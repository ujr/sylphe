% Griechisch mit Plain TeX
% ujr/2020-04-21

\magnification\magstep1

% fixpdfmag: pdftex applies magnification even to page dimensions - undo!
\ifx\pdfpageheight\undefined\else
  \message{Fixing pdftex magnification annoyance.}
  \begingroup \catcode`\@=11
  \global\pdfpageheight\expandafter\getf@ctor\the\pdfpageheight truept
  \global\pdfpagewidth\expandafter\getf@ctor\the\pdfpagewidth truept
  \global\pdfhorigin\expandafter\getf@ctor\the\pdfhorigin truept
  \global\pdfvorigin\expandafter\getf@ctor\the\pdfvorigin truept
  \endgroup
\fi

\input german.sty

\ifx\pdfoutput\undefined\csname newcount\endcsname\pdfoutput\fi
\ifcase\pdfoutput\def\pdfinfo#1{\special{pdf:docinfo<<#1>>}}\fi

\pdfinfo{ /Author (Urs-Jakob Ruetschi)
  /Title (Griechisch mit Plain TeX)
  /Subject (Greek alphabet, diacritics, and TeX) }

\hsize=15.0truecm \advance\hoffset 4.6 truemm % (21-15)/2 - 2.54
\vsize=23.0truecm \advance\voffset 8.1 truemm % (29.7-23)/2 - 2.54

\parskip=0pt plus 2pt
\parindent=18pt
\baselineskip=13pt plus .4pt minus .2pt
\hfuzz=1pt % lines may stick a little into the right margin

\font\tf=cmbx17 % title
\font\sf=cmbx12 % section headings
\font\sc=cmr9 % small caps relative to cmr10
\font\eightrm=cmr8
\font\sevenrm=cmr7
\font\ninett=cmtt9 % for listings

\def\section #1\par{\vskip 0pt plus .1\vsize\penalty-250
  \vskip 0pt plus -.1\vsize \vskip 24pt plus 6pt minus 4pt
  \vskip\parskip \centerline{\sf #1}%
  \nobreak\medskip\noindent\ignorespaces}%

\def\signed #1\par{{\unskip\nobreak\hfil\penalty50
  \hskip2em\hbox{}\nobreak\hfil\sl#1\relax
  \parfillskip=0pt \finalhyphendemerits=0 \par}}

\def\gr{\greekmode}% Usage: {\gr ...}
\def\vChr{\unskip\hskip.67ex v.\kern.5ex Chr.}%
\def\nChr{\unskip\hskip.67ex n.\kern.5ex Chr.}%
\def\zB{z.\kern.3ex B.}% zum Beispiel
\def\dh{d.\kern.3ex h.}% das heisst
\def\LaTeX{L\kern-.7ex\raise.4ex\hbox{\sevenrm A}\kern-.3ex\TeX}%
\def\*{\smallskip\hangindent\parindent\hangafter1\noindent\ignorespaces}%
\def\\{\hfil\break}% newline
\def\nex#1{\kern-.#1ex}% negative kern by 0.#1 ex
\def\bull{\vrule width .5ex height .8ex depth -.3ex}% bullet

%=== Verbatim listing ==========================================
\def\verbatim{\begingroup
 \def\do##1{\catcode`##1=12 }\dospecials
 \parskip 0pt \parindent 0pt \baselineskip10pt \ninett
 \catcode`\ =13 \catcode`\^^M=13 % active
 \catcode`\?=0}
{\catcode`\^^M=13{\catcode`\ =13
  \gdef\verbatimdefs{\def^^M{\ \par}\let =\ }}%
  \gdef\verbatimgobble#1^^M{}}%
\def\beginverbatim{\smallskip\begingroup\verbatim}
\def\endverbatim{\endgroup\endgroup\smallskip}

% \obeyspaces \obeylines \tt}
%{\obeyspaces\gdef {\ }} % \obeyspaces now gives \ , not \space

%\catcode`\?=\active % our replacement escape
%\newskip\ttglue \ttglue=.5em plus.25em minus.15em
%{\obeylines\gdef?{\ttverbatim\spaceskip=\ttglue%
% \let^^M=\ \let?=\endgroup}}
%===============================================================

%=== Greek names not already defined in plain TeX ==============
% Mathcode: xyzz where x=class, y=family, zz=fontpos (all hex)
% Class:    0=ordinary, ..., 7=extensible
% Family:   0=roman, 1=math italic, 2=math symbol, 3=math extension
\mathchardef\omicron="016F  % o
% in the following, use family 1 (eg, "0141) for slanted
\mathchardef\Alpha="0041    % A
\mathchardef\Beta="0042     % B
\mathchardef\Epsilon="0045  % E
\mathchardef\Zeta="005A     % Z
\mathchardef\Eta="0048      % H
\mathchardef\Iota="0049     % I
\mathchardef\Kappa="004B    % K
\mathchardef\Mu="004D       % M
\mathchardef\Nu="004E       % N
\mathchardef\Omicron="004F  % O
\mathchardef\Rho="0050      % P
\mathchardef\Tau="0054      % T
\mathchardef\Chi="0058      % X
%===============================================================

%=== from Levy's greekmacros, reformatted  =====================
\font\tengreek=grreg10 \font\tengrbf=grbld10 % load fonts
%
\def\greekmode{% change category codes for diacritics
 \catcode`\<=13 \lccode`\<=`\<% make active (defined below)
 \catcode`\>=13 \lccode`\>=`\>%
 \catcode`\'=11 \lccode`\'=`\'% make letter (ligatures in font)
 \catcode`\`=11 \lccode`\`=`\`%
 \catcode`\~=11 \lccode`\~=`\~%
 \catcode`\"=11 \lccode`\"=`\"%
 \catcode`\|=11 \lccode`\|=`\|%
 \tengreek\def\bf{\tengrbf}}
% Save old meanings of breathing marks:
\def\lt{<}
\def\gt{>}
% Helper macros (to see what follows a breathing mark):
\def\ifnextchar#1#2#3{\let\tempe #1\def\tempa{#2}\def\tempb{#3}\futurelet
  \tempc\ifnch}% if next char is #1 then expand to #2 else #3
\def\ifnch{\ifx\tempc\tempe\let\tempd\tempa\else\let\tempd\tempb\fi\tempd}
\def\gobble#1{}% just remove the next token
% Vowel (next char) and accent:
\newcount\vwl
\newcount\acct
{
  \greekmode
  \gdef>{\ifnextchar `{\expandafter\smoothgrave\gobble}{\char\lq\>}}
  \gdef<{\ifnextchar `{\expandafter\roughgrave\gobble}{\char\lq\<}}
  \gdef\smoothgrave#1{\acct=\rq137 \vwl=\lq#1 \dobreathinggrave}
  \gdef\roughgrave#1{\acct=\rq103 \vwl=\lq#1 \dobreathinggrave}
  \gdef\dobreathinggrave{\ifnum\vwl\lt\rq140    %if uppercase
    \char\the\acct\char\the\vwl\else\expandafter\testiota\fi}
  \gdef\testiota{\ifnextchar |{\addiota\doaccent\gobble}{\doaccent}}
  \gdef\addiota{\ifnum\vwl=\lq a\vwl=\rq370
    \else\ifnum\vwl=\lq h\vwl=\rq371 \else\vwl=\rq372 \fi\fi}
  \gdef\doaccent{\accent\the\acct \char\the\vwl\relax}
}
%===============================================================


\topglue 2pc plus 1filll

\centerline{\tf Griechisch mit Plain \TeX}
\nobreak\bigskip\noindent
{\sl Vorbemerkung:} weder kann ich griechisch, noch bin ich philologisch
gebildet; die Ausf"uhrungen hier sind sicher unvollst"andig und
m"oglicherweise auch falsch. Sie halfen mir aber immer wieder, ein
griechisches Wort mit Plain \TeX\ "uberhaupt schreiben zu k"onnen.
F"ur mehr als einige Worte sind die Methoden hier nicht geeignet,
und f"ur \LaTeX\ sind andere Quellen zu konsultieren.
\signed ---ujr


\section Alphabet

Die Griechen haben um 800\vChr\ das ph"onizische Alphabet "ubernommen
und verbessert; insbesondere wurden Zeichen f"ur die Vokale erg"anzt.
Es umfasst 24 Buchstaben. Die Minuskeln (Kleinbuchstaben) wurden erst
ab 500\nChr\ entwickelt. Eine Besonderheit ist das kleine Sigma, welches
innerhalb des Wortes als $\sigma$, am Ende des Wortes aber als $\varsigma$
geschrieben wird, \zB\ {\gr fil'osofos}. % philosophos

$$\vbox{\offinterlineskip
\tabskip 1em plus 1em
\def\loosen{\omit&\omit& height 2pt&
  \omit&\omit& height 2pt& \omit&\omit& height2pt&\omit\cr}
\halign{\strut\hfil\eightrm#& #\hfil& \vrule#&
  \hfil$#$\hfil& \hfil$#$\hfil& \vrule#&
  \hfil{\gr#}\hfil& \hfil{\gr#}\hfil& \vrule#&
  \hfil#\hfil\cr
\omit Buchstabe\hfil\span&& \omit Knuth\span&& \omit Levy\span&& Aussprache\cr
\loosen
\noalign{\hrule}
\loosen
 1& Alpha&&   \Alpha&   \alpha&&   A& a&& a\cr
 2& Beta&&    \Beta&    \beta&&    B& b&& b\cr
 3& Gamma&&   \Gamma&   \gamma&&   G& g&& g\cr
 4& Delta&&   \Delta&   \delta&&   D& d&& d\cr
 5& Epsilon&& \Epsilon& \varepsilon,\epsilon&& E& e&& e {\eightrm(kurz)}\cr
 6& Zeta&&    \Zeta&    \zeta&&    Z& z&& z\cr
 7& Eta&&     \Eta&     \eta&&     H& h&& e, "a\cr
 8& Theta&&   \Theta&   \theta,\vartheta&&   J& j&& th\cr
 9& Iota&&    \Iota&    \iota&&    I& i&& i, j\cr
10& Kappa&&   \Kappa&   \kappa&&   K& k&& k\cr
11& Lambda&&  \Lambda&  \lambda&&  L& l&& l\cr
12& My&&      \Mu&      \mu&&      M& m&& m\cr
13& Ny&&      \Nu&      \nu&&      N& n&& n\cr
14& Xi&&      \Xi&      \xi&&      X& x&& x\cr
15& Omikron&& \Omicron& \omicron&& O& o&& o {\eightrm(kurz)}\cr
16& Pi&&      \Pi&      \pi&&      P& p&& p\cr
17& Rho&&     \Rho&     \rho,\varrho&&     R& r&& r, rh\cr
18& Sigma&&   \Sigma&   \sigma,\varsigma&& S& \char99, s&& s\cr
19& Tau&&     \Tau&     \tau&&     T& t&& t\cr
20& Ypsilon&& \Upsilon& \upsilon&& U& u&& "u, y\cr
21& Phi&&     \Phi&     \phi,\varphi&&     F& f&& f, ph\cr
22& Chi&&     \Chi&     \chi&&     Q& q&& ch\cr
23& Psi&&     \Psi&     \psi&&     Y& y&& ps\cr
24& Omega&&   \Omega&   \omega&&   W& w&& o {\eightrm(lang)}\cr
}}$$

\eject

\noindent
Die angegebene Aussprache bezieht sich auf das Altgriechische;
das Neugriechische weicht zum Teil stark davon ab.
In der Spalte ``Knuth'' sind die Buchstaben aus den Computer Modern
({\sc CM}) Fonts von D.\kern.5ex E.\kern.5ex Knuth dargestellt,
in der Spalte ``Levy'' jene von Silvio Levy, welche von den
{\sc CM} Fonts abgeleitet wurden. Die Grossbuchstaben sind sich
"ahnlich; Knuths Minuskeln sind kursiv, Levys aufrecht, aber
mit einer geneigten Achse im Unterschied zur vertikalen Achse
der lateinischen Buchstaben in {\sc CM} Roman.
Levys Font ist ein Pixelfont, {\sc CM} liegt heute allgemein
als Vektorfont vor.


\section Diphthonge

Diphthonge sind Doppellaute aus zwei verschiedenen Vokalen innerhalb
einer Silbe, \zB\ h\underbar{eu}te.
Sie sind zu unterscheiden vom Hiatus, welcher das Aufeinandertreffen
zweier Vokale verschiedener Silben bezeichnet,
\zB\ R\underbar{u$\cdot$i}ne.
Das Altgriechische kennt folgende Diphthonge:

$$\vbox{\offinterlineskip
\def\loosen{\omit&\omit& height 2pt&\omit\cr}%
\tabskip 1em
\halign{\tabskip 1em
 \strut\hfil$#$\hfil& \hfil$#$\hfil& \vrule#& \hfil#\hfil\cr
 \span\omit\hfil Diphthong\hfil&&\omit\hfil Aussprache\hfil\cr
 \loosen
 \noalign{\hrule}
 \loosen
 \Alpha\Upsilon&   \alpha\upsilon&      & au\cr
 \Omicron\Upsilon& \omicron\upsilon&    & u\cr
 \Omicron\Iota&    \omicron\iota&       & oi\cr
 \Epsilon\Iota&    \varepsilon\iota&    & e$\cdot$i\rlap{\enspace(!)}\cr
 \Alpha\Iota&      \alpha\iota&         & ai\cr
 \Epsilon\Upsilon& \varepsilon\upsilon& & oi, eu\cr
}}$$


\section Diakritische Zeichen

Mit Diakritika wird die altgriechische Aussprache angezeigt.
Es gibt drei Akzente f"ur die Tonlage, zwei Zeichen f"ur Hauchlaute,
das Iota subscriptum, das Trema und weitere.

\medskip
\item{--}
Akut ({\gr >oxe~ia}, engl.~{\it acute\/}):
helle Tonlage, Steigton
\item{--}
Gravis ({\gr bare~ia}, {\it grave\/}):
dumpfe Tonlage, Fallton
\item{--}
Zirkumflex ({\gr perispwm'enh}, {\it circumflex}):
Zusammenzug aus Akut und Gravis
\medskip
\item{--}
Spiritus asper ({\gr dase~ia}, {\it rough breathing\/}):
rauher = h"orbarer Hauch, \zB\ {\gr <istor'ia} % historia
\item{--}
Spiritus lenis ({\gr yil'otes, yil'h}, {\it smooth breathing\/}):
sanfter = stummer Hauch, zeigt an, dass {\it kein\/} Hauch vorliegt,
\zB\ {\gr >'\nex2Erws, >ast'hr} % Eros, aster
\medskip
\item{--}
Iota subscriptum: ein kleines Iota unter {\gr a, h, w},
also {\gr a|, h|, w|}, \zB\ {\gr tragw|d'ia} % tragodia
\item{--}
Iota adscriptum: bei Majuskeln wird es rechts angeh"angt,
\zB\ {\gr <'\nex6 Aidhs, >\nex4 Oide~ion} % Hades, Oideion
\medskip
\item{--}
Trema ({\gr tr~hma}, {\gr dialutik'a}, engl.~{\it diaeresis\/}):
zeigt an, dass zwei Vokale {\it keinen\/} Diphthong bilden, dass
also ein Hiatus vorliegt, \zB\ {\gr >\nex7 Atre"'idhs} (vier Silben)

\medskip\noindent
Die Diakritika stehen {\it "uber\/} Minuskeln, aber {\it vor\/} Majuskeln,
\zB\ {\gr <\nex3Ell'as}. % Hellas

Akut und Gravis k"onnen auf allen Vokalen stehen,
Zirkumflex auf allen langen Vokalen (also nicht auf $\varepsilon$
und $\omicron$, die immer kurz sind).
Die Hauchzeichen k"onnen auf Vokalen und $\rho$ stehen.
Ein $\upsilon$ am Wortanfang tr"agt immer ein Spiritus asper.
Ein Akzent kann mit einem Hauchzeichen oder dem Trema auf
einem Buchstaben koexistieren, \zB\ {\gr >'onoma}. % onoma

Griechische W"orter sind auf einer der drei letzten Silben betont.
Der Akut kann eine Betonung auf jeder dieser Silben anzeigen.
Wenn die Betonung auf der letzten Silbe liegt und ohne Satzzeichen
ein weiteres Wort folgt, wird ein Gravis gesetzt.
Der Zirkumflex kann nur auf den beiden letzten Silben gesetzt werden.
Alle diese Varianten haben Namen, z.B. Ox\'ytonon f"ur einen Akut
(oder eine Gravis) auf der letzten Silbe; sie sind hier nicht
weiter aufgef"uhrt.

Griechenland hat 1982 die vereinfachte ``monotonische''
Orthographie eingef"uhrt, welche nur noch ein diakritisches
Zeichen verwendent, den {\gr t'onos}, um die betonte Silbe
anzuzeigen; er wird wie der Akut geschrieben.

\bigbreak\noindent
{\sl Anmerkungen:\/} das Trema ist vom Umlaut zu unterscheiden,
obschon beide gleich dargestellt werden.
\bull\ Die Herkunft des Spiritus lenis ist ungewiss.
\bull\ In Unicode sind die Kombinationen aus Buchstaben und Diakritika
als eigene Zeichen codiert, \zB\ steht {\tt U+1F02} f"ur {\gr >`a}.
Die klassische \TeX-Engine kann mit Unicode Fonts nicht
umgehen, weil sie maximal 256 Glyphen pro Font erlaubt.
Neuere Unicode-f"ahige \TeX-Erweiterungen sind Xe\TeX\ und Lua\TeX,
diese sind hier aber nicht behandelt.


\section Computer Modern Fonts

Plain \TeX\ ({\tt plain.tex}) baut auf den Computer Modern ({\sc CM})
Fonts auf. Diese enthalten auch Glyphen f"ur die griechischen Buchstaben,
soweit sie sich von den lateinischen unterscheiden. Somit stehen auf
allen \TeX-Systemen griechische Buchstaben zur Verf"ugung, allerdings
nur im Mathematik-Modus. Beispiele:
$$
\vbox{\settabs\+\hskip3cm&\cr
\let\s=\string
\+$a\sin\beta=b\sin\alpha$&
 {\tt\$a\string\sin\string\beta=b\string\sin\string\alpha\$}\cr
\+$\bar\xi={1\over n}\sum \xi_i$&
 {\tt\$} ...{\tt\s\sum\s\xi\_i\$}\qquad({\tt\s\sum} produces a large sigma)\cr
\+$\chi\alpha\acute\iota\rho\epsilon\theta\epsilon$&
 \tt\$\s\chi\s\alpha\s\acute\s\iota\s\rho\s\epsilon\s\theta\s\epsilon\$\cr
\+$\rm XAIPE\Theta E$&
 \tt\$\s\rm\ XAIPE\s\Theta\ E\$\cr
\+$\it XAIRE\Theta E$&
 \tt\$\s\it\ XAIRE\s\Theta\ E\$\qquad\eightrm(seid gegr"usst)\cr
}
$$
Die Minuskeln stehen nur kursiv zur Verf"ugung, die Majuskeln
aber aufrecht und kursiv. Plain \TeX\ definiert Steuerbefehle
nur f"ur jene griechischen Buchstaben, welche sich von den
lateinischen unterscheiden; die ``fehlenden'' liessen sich,
falls gew"unscht, einfach erstellen.
Akzente ({\tt\string\acute}, {\tt\string\grave}, {\tt\string\hat})
und Trema ({\tt\string\ddot}) auf Kleinbuchstaben sind unmittelbar
m"oglich, danach wird es schwieriger.

Was f"ur den mathematischen Formelsatz intuitiv und effizient
ist, wird f"ur ganze griechische W"orter oder gar Texte bald
m"uhsam. Es w"are m"oglich, sich auf Basis der Glyphen in den
{\sc CM} Fonts geeignete Macros zu bauen, aber auch damit w"aren
keine nicht-kursiven Minuskeln m"oglich. Es braucht griechische
Fonts.


\section Levy's Greek Fonts

Ende der 1980er hat Silvio Levy von {\sc CM} griechische Fonts
abgeleitet ({\tt grreg}, {\tt grbld}, {\tt grtt}) und passende
Makros ({\tt greekmacros.tex}) erstellt, welche
"uber das Paket {\tt levy-font} auf {\sc CTAN} zur Verf"ugung
stehen und wohl in den meisten \TeX-Distributionen ohnehin
enthalten sind.

Die Verwendung erfolgt nach {\tt\string\input~greekmacros}
durch {\tt\char"7B\string\greekmode~abg\char"7D}
oder {\tt\string\begingreek\ Greek text\string\endgreek},
wobei der griechische Text mit lateinischen Buchstaben
gem"ass folgender Tabelle einzugeben ist:
$$\hbox{%
 \valign{\hbox to 11pt{\hfil\strut{\gr#}\hfil}&
         \hbox to 11pt{\hfil\strut\tt#\hfil}\cr
 a&a\cr b&b\cr g&g\cr d&d\cr e&e\cr z&z\cr h&h\cr j&j\cr
 i&i\cr k&k\cr l&l\cr m&m\cr n&n\cr x&x\cr o&o\cr p&p\cr
 r&r\cr c&s\cr t&t\cr u&u\cr f&f\cr q&q\cr y&y\cr w&w\cr}}
$$
$$\hbox{%
 \valign{\hbox to 11pt{\hfil\strut{\gr#}\hfil}&
         \hbox to 11pt{\hfil\strut\tt#\hfil}\cr
 A&A\cr B&B\cr G&G\cr D&D\cr E&E\cr Z&Z\cr H&H\cr J&J\cr
 I&I\cr K&K\cr L&L\cr M&M\cr N&N\cr X&X\cr O&O\cr P&P\cr
 R&R\cr S&S\cr T&T\cr U&U\cr F&F\cr Q&Q\cr Y&Y\cr W&W\cr}}
$$
Ein {\tt s} ergibt je nach Position im Wort ein $\sigma$
oder ein $\varsigma$, ein {\tt c} immer ein $\sigma$.
%
Akzente werden mit vorangestelltem {\tt\char"27} (Akut),
{\tt\char"60} (Gravis), {\tt\char"7E} (Zirkumflex) erstellt,
Hauchzeichen mit vorangestelltem {\tt\char"3C} (rauh),
{\tt\char"3E} (stumm), das Trema mit vorangestelltem~{\tt\char"22},
und das Iota subscriptum mit nachgestelltem~{\tt\char"7C}.
Beispiele:
$$\vbox{\originalTeX\catcode`\~=11
 \halign{\strut#\hfil&\qquad{\tt#}\hfil\cr
  {\gr\spaceskip.5em 'a `a ~a <a >a <'a >`a >~a| "'i}
  & 'a `a ~a <a >a <'a >`a >~a| "'i\cr
  {\gr >en >arq~h| >~hn <o l'ogos}
  & >en >arq~h| >~hn <o l'ogos\cr
}}
$$
Die Akzente sind "uber Ligaturen in den Fonts realisiert,
die Hauchzeichen je nach Kontext auch mit {\tt\string\accent}.
Das Kerning zwischen Akzenten und Grossbuchstaben ist
nicht optimal; in diesem Text wurde oft mittels expliziten
Unterschneidungen (\zB\ {\tt\string\kern-.3ex}) nachgeholfen.
Das Trema scheint nur auf {\gr i} und {\gr u} zu gehen.

Levys Fonts enthalten die folgenden Satzzeichen, wobei
{\bf;} unserem {\bf?} entspricht, und
der Apostroph und die Anf"uhrungszeichen "uber Ligaturen
zu erreichen sind:
$$\hbox{\offinterlineskip%
 \valign{\hbox to 18pt{\hfil\strut{\gr#}\hfil}%
         \hbox to 18pt{\vrule width0pt height0pt depth 2pt % line indicates
                       \leaders\hrule height .1pt\hfill}&  % pos of punct.
         \hbox to 18pt{\hfil\tt#\strut\hfil}\cr
 .&.\cr ,&,\cr ;&;\cr :&:\cr !&!\cr ?&?\cr ''&''\cr ((&((\cr ))&))\cr
}}$$

Es stehen die Fonts {\tt grreg} (regular) und {\tt grbld} (bold)
je in den Punktgr"ossen 10, 9, 8 zur Verf"ugung.
"Uber die Makros {\tt\string\tengr} und {\tt\string\tengrbf}
sind die 10-Punkt-Schriften erreichbar.
{\bf Achtung:} {\tt greekmacros.tex} l"adt auch den Font {\tt grtt10},
welcher in meiner Erfahrung aber in \TeX-Distributionen
nicht zur Verf"ugung steht. Entweder ab Levys Quellen bilden und
installieren, oder {\tt greekmacros.tex} entsprechend bearbeiten,
am besten eine Kopie im eigenen {\tt\char"7E/texmf/tex} Verzeichnis.


\section Beispiele

\settabs\+\hskip1em&\hskip10pc&\cr
\+&{\gr\llap{((}E<'ureka, e<'ureka!))}&
 Heureka, ich habe es gefunden! (Archimedes)\cr
\+&{\gr >apoj'ewsis}&
 Apotheose, Verherrlichung, ``Vergottung''\cr
\+&{\gr <or'izwn k'uklos}&
  umrandender Kreis, Horizont\cr
\+&{\gr Sf'igx, sugkop'h}&
 Sphinx, Synkope ({\gr g} vor {\gr g k x q} als {\it n} gesprochen)\cr
\+&{\gr Gewgrafik`h <\nex4 Uf'hghsis}&
  Geographische Kartenlehre (des Ptolem"aus)\cr
\+&{\gr YUQHS IA\nex3TR\nex1EION}&
  ``Seelenapotheke'' (Bibliotheksinschrift)\cr
\+&{\gr\llap{>\nex3}En >arq~h| >~hn <o l'ogos}&
  Im Anfang war das Wort (Joh 1:1)\cr


\section Makro-Notizen

Levys {\tt\string\greekmode} Makro macht {\tt\char"27}
{\tt\char"60} {\tt\char"7E} {\tt\char"22} {\tt\char"7C}
gew"ohnliche Buchstaben ({\tt\string\catcode}~11), denn
diese Akzente sind in den Fonts als Ligaturen hinterlegt;
{\tt\char"3C} und {\tt\char"3E} werden aktiv gemacht
({\tt\string\catcode}~13) und pr"ufen, ob ein {\tt\char"60}
folgt; diese Kombination pr"uft weiter, ob ein Gross- oder
Kleinbuchstabe folgt, und allenfalls ein Iota-Subskript,
und expandiert entweder als gew"ohnliche Buchstaben oder
ein {\tt\string\accent}-Konstrukt. Grund f"ur diese
Komplikation ist, dass die maximal 256 Slots in einem
Font nicht reichen f"ur alle Kombination, weswegen
die seltene Hauchzeichen-Gravis-Kleinbuchstabe-Kombination
mittels {\tt\string\accent} zusammengesetzt wird.
Schliesslich wird mit {\tt\string\tengr} zum Font {\tt grreg10}
gewechselt.

Makros und Fonts sind also aufs engste verkn"upft.
Das ist ganz anders mit {\tt german.sty}, wo einzig
{\tt\char"22} aktiv gemacht wird und die deutschen
Umlaute aus den Glyphen in {\sc CM} zusammengesetzt
werden.

\TeX\ weist seine Category Codes nur beim ersten Lesen zu,
also schon beim Scannen von Makro-Argumenten, nicht erst bei
der Makro-Expansion. Das kann bei den {\tt greekmacros}
zu unerwarteten Effekten f"uhren, weil die {\tt\string\catcode}s
ja erst im {\tt\string\greekmode} angepasst werden;
bei {\tt german.sty} ist das kein Problem, weil hier der
{\tt\string\catcode} von {\tt\char"22} einmalig beim Laden
der Makros ge"andert wird (Anwendungen von {\tt\string\originalTeX}
vorbehalten).

%\medbreak\noindent
Wie schon erw"ahnt hat Plain \TeX\ keine Steuerbefehle f"ur
griechische Buchstaben, die formgleich mit lateinischen sind.
Die ``fehlenden'' Steuerbefehle sind:
\beginverbatim
\mathchardef\Alpha="0041   \mathchardef\Iota="0049  \mathchardef\Rho="0050
\mathchardef\Beta="0042    \mathchardef\Kappa="004B \mathchardef\Tau="0054
\mathchardef\Epsilon="0045 \mathchardef\Mu="004D    \mathchardef\Chi="0058
\mathchardef\Zeta="005A  \mathchardef\Nu="004E  \mathchardef\omicron="016F
\mathchardef\Eta="0048   \mathchardef\Omicron="004F
?endverbatim
\noindent
W"urde man bei den Grossbuchstaben die zweite {\tt 0} durch eine {\tt 1}
ersetzen (\dh\ Font\-familie~1, also \zB\ {\tt\char"22 0141}),
erhielte man kursive Buchstaben. Dabei w"aren nat"urlich auch die
schon in {\tt plain.tex} enthaltenen Definitionen zu "uberschreiben.


\section Referenzen

\raggedright

\*
Silvio Levy, Using Greek Fonts with \TeX, {\sl TUGboat} {\bf 9}:1, 1988,\\
{\tt http://tug.org/TUGboat/tb09-1/tb20levy.pdf}

\*
{\tt levy-font}: das Paket auf {\sc CTAN} mit Levys Fonts und Makros:\\
{\tt https://ctan.org/pkg/levy-font}

\*
D.\kern.5ex E.\kern.5ex Knuth, {\sl The \TeX book}, Addison-Wesley,
1984, 1986, 1996;\\
darin insbesondere Anhang~F mit den {\sc CM} Font-Tabellen.

\*
Wikipedia: Polytonische Orthographie (Zugriff am 23.~April~2020)\\
{\tt https://de.wikipedia.org/wiki/Polytonische\_Orthographie}

\*
Unicode: Character Code Charts online at\\
{\tt https://www.unicode.org/charts/PDF/U0370.pdf} (Greek and Coptic)\\
{\tt https://www.unicode.org/charts/PDF/U1F00.pdf} (Greek Extended)

\*
Claudio Beccari, The teubner \LaTeX\ package: Typesetting
classical Greek philology, {\sl TUGboat} {\bf 23}:3/4, 2002,\\
{\tt https://www.tug.org/TUGboat/tb23-3-4/tb75beccteub.pdf}

\bye

