% Jules Verne, 1879: Les Révoltés de la Bounty

\magnification=\magstep2

\input fonts
\input epsf

\ifx\pdfoutput\undefined\csname newcount\endcsname\pdfoutput\fi
\ifcase\pdfoutput\def\pdfinfo#1{\special{pdf:docinfo<<#1>>}}\fi

\pdfinfo{ /Author (Jules Verne 1879)
 /Title (Les R^^e9volt^^e9s de la Bounty) 
 /Subject (Typeset from public sources by U.J.R.) }

\hsize=15truecm \advance\hoffset 5truemm
\vsize=23truecm

\parskip=0pt plus 1pt
\parindent=1pc

% Very flexible interword glue and not very fuzzy about right margin:
\spaceskip=.3333em plus .22em minus .18em
\hfuzz=1pt

\font\tf=cmbx17
\font\sf=cmbx12
\font\sc=cmr9 % small caps, relativ zu cmr10

\def\fig#1{\hbox to 10cm{\special{pdf:image width 10cm (#1)}\hss}}%
\def\fn#1#2{\footnote{\hbox{\eightpoint#1}}{\eightpoint#2}}%
\let\,\thinspace

% Could use HyperTex \specials {html:<a href="...">} and {html:</a>}
% but dvipdfm then generates ugly blue boxes around the link text;
% should check pdfTeX capabilities... (but pdfTeX gives larger PDF files).
\def\href#1#2{{#2}}%

\def\sylphech{\leavevmode\href{http://www.sylphe.ch}{www.sylphe.ch}}%
\def\CCBYSA{\leavevmode\href
  {https://creativecommons.org/licenses/by-sa/4.0/}{\sc CC BY\kern-.3ex -SA}}%
\def\copytext{2020 $\cdot$ \sylphech\ $\cdot$ \CCBYSA}%

\catcode`|=\active\def|{\kern.4ex}% 

% Definitions to help with French punctuation spacing:
% guillemets, space before : ! ? and inside << and >>

\font\sy=cmcyr9 % has guillemets << and >> at code 29 and 30
\def\ql{\leavevmode\raise.1ex\hbox{\sy\char 29}\kern.4ex}%
\def\qr{\kern.4ex\raise.1ex\hbox{\sy\char 30}}%
\def\colon{\kern.4ex:}\def\exclam{\kern.4ex!}\def\questm{\kern.4ex?}%

\begingroup\catcode`<=\active\catcode`>=\active
  \catcode`:=\active\catcode`?=\active\catcode`!=\active
\gdef\french{% make active and assign kerning/punctuation
  \catcode`<=\active \let<=\ql
  \catcode`>=\active \let>=\qr
  \catcode`:=\active \let:=\colon
  \catcode`!=\active \let!=\exclam
  \catcode`?=\active \let?=\questm
}\gdef\endfrench{% make them ordinary "other" chars again
  \catcode`<=12\catcode`>=12\catcode`:=12\catcode`!=12\catcode`?=12
}\endgroup

% Titelei:
% 1 Schmutztitel
% 2 vakat
% 3 Titelseite
% 4 Impressum

\begingroup
\nopagenumbers


\topglue5pc
\centerline{\tf Les R\'evolt\'es de la Bounty}
\bigskip
\centerline{Jules Verne, 1879}
\vfill\eject


\topglue5pc
% vakat
\vfill\eject


\topglue5pc
\centerline{\tf Les R\'evolt\'es de la Bounty}
\bigskip
\centerline{par {\bf Jules Verne},}
\smallskip
\centerline{avec cinq illustrations de S. Dr\'ee,}
\smallskip
\centerline{publi\'e par Jules Hetzel \`a Paris en {\bf 1879}.}
\vfill
\begingroup\french\narrower\noindent\it
Nous croyons bon de pr\'evenir nos lecteurs que ce r\'ecit
n'est point une fiction. Tous les d\'etails en sont pris aux
annales maritimes de la Grande-Bretagne. La r\'ealit\'e fournit
quelquefois des faits si romanesques que l'imagination elle-m\^eme
ne pourrait rien y ajouter.
\par\endgroup
\vfill\eject


\topglue5pc
\vfill
\begingroup\ninepoint
\noindent
Der vorliegende Text ist eigentlich von Gabriel Marcel (1843--1909),
einem Geo\-graphen der Biblioth\`eque National, welcher mit
Jules Verne (1828--1905) und seinem Verleger Pierre-Jules
Hetzel (1814--1886) zu\-sam\-men\-ar\-bei\-te\-te.
Verne hat den Text gegengelesen und f\"ur 300 Francs alle
Rechte an der Kurzgeschichte erworben. Er erschien 1879
als Anhang zum Roman {\it Les Cinq Cents Millions de la B\'egum}
in der Reihe {\it Voyages Extraordinaires}, welche aus insgesamt
68 Romanen und Kurzgeschichten von Verne besteht und
zwischen 1863 und 1919 (ab 1905 posthum) erschien.
\smallskip\noindent
Die Kurzgeschichte im Umfang von etwa 8000 W\"ortern
ist in drei Kapitel gegliedert: {\it die Aussetzung}, {\it die
Ausgesetzten\/} und {\it die Aufst\"andigen}. Dabei nimmt
die Leistung des ausgesetzten Kapit\"ans William Bligh und
der ihm Loyalen einen
besonderen Stel\-len\-wert ein, dies im Gegensatz zu anderen
Erz\"ahlungen der Meuterei auf der {\sl Bounty}. 
\smallskip\noindent
\settabs\+\hskip5.5em&\cr
\+Textquelle:&Wikisource, die freie Quellensammlung\cr
\+Bildquelle:&jv.gilead.org.il/rpaul/\cr
\+Textsatz, Karte, Anhang: {\sc UJR}, im Fr\"uhjahr 2020\cr
\bigskip
\endgroup
\centerline{\eightpoint \copytext}
\eject

\endgroup
\french


\topglue 5pc
\centerline{\sf I}
\bigskip
\centerline{\sf L'ABANDON}
\vskip 2pc
\noindent
Pas le moindre souffle, pas une ride \`a la surface de
la mer, pas un nuage au ciel. Les splendides constellations
de l'h\'emisph\`ere austral se dessinent avec une
incomparable puret\'e.
Les voiles de la {\sl Bounty\/} pendent le long des m\^ats,
le b\^atiment est immobile, et la lumi\`ere de la lune,
p\^alissant devant l'aurore qui se l\`eve,
\'eclaire l'espace d'une lueur ind\'efinissable. 

La {\sl Bounty}, navire de deux cent quinze tonneaux
mont\'e par quarante-six hommes, avait quitt\'e Spithead,
le 23~d\'ecembre 1787, sous le commandement du capitaine Bligh,
marin exp\'eriment\'e mais un peu rude, qui avait accompagn\'e
le capitaine Cook dans son dernier voyage d'exploration.

La {\sl Bounty\/} avait pour mission sp\'eciale de transporter
aux Antilles l'arbre \`a pain, qui pousse \`a profusion dans
l'archipel de Tahiti. Apr\`es une rel\^ache de six mois dans
la baie de Matava\"\i, William Bligh, ayant charg\'e un millier
de ces arbres, avait pris la route des Indes occidentales,
apr\`es un assez court s\'ejour aux {\^\i}les des Amis.

Maintes fois, le caract\`ere soup\c{c}onneux et emport\'e
du capitaine avait amen\'e des sc\`enes d\'esagr\'eables
entre quelques-uns de ses officiers et lui. Cependant,
la tranquillit\'e qui r\'egnait \`a bord de la {\sl Bounty\/},
au lever du soleil, le 28~avril 1789, ne faisait rien
pr\'esager des graves \'ev\'enements qui allaient se produire.

Tout semblait calme, en effet, lorsque tout \`a coup
une animation insolite se propage sur le b\^atiment.
Quelques matelots s'accostent, \'echangent deux ou trois
paroles \`a voix basse, puis disparaissent \`a petits pas.

Est-ce le quart du matin qu'on rel\`eve?
Quelque accident inopin\'e s'est-il produit \`a bord?

<Pas de bruit surtout, mes amis, dit Fletcher Christian,
le second de la {\sl Bounty}. Bob, armez votre pistolet,
mais ne tirez pas sans mon ordre. Vous, Churchill, prenez
votre hache et faites sauter la serrure de la cabine du capitaine.
Une derni\`ere recommandation: Il me le faut vivant!>

Suivi d'une dizaine de matelots arm\'es de sabres,
de coutelas et de pistolets, Christian se glissa dans
l'entrepont; puis, apr\`es avoir plac\'e deux
sentinelles devant la cabine de Stewart et de Peter Heywood,
le ma{\^\i}tre d'\'equipage et le midshipman de la
{\sl Bounty}, il s'arr\^eta devant la porte du capitaine.

<Allons, gar\c{c}ons, dit-il, un bon coup d'\'epaule!>

La porte c\'eda sous une pression vigoureuse,
et les matelots se pr\'ecipit\`erent dans la cabine.

Surpris d'abord par l'obscurit\'e,
et r\'efl\'echissant peut-\^etre
\`a la gravit\'e de leurs actes,
ils eurent un moment d'h\'esitation.

<Hol\`a! qu'y a-t-il?
Qui donc ose se permettre? ...> s'\'ecria
le capitaine en sautant \`a bas de son cadre.

--- Silence, Bligh! r\'epondit Churchill. Silence,
et n'essaie pas de r\'esister, ou je te b\^aillonne!

--- Inutile de t'habiller, ajouta Bob. Tu feras toujours
assez bonne figure, lorsque tu seras pendu \`a la
vergue d'artimon!

--- Attachez-lui les mains derri\`ere le dos, Churchill,
dit Christian, et hissez-le sur le pont!

--- Le plus terrible des capitaines n'est pas bien redoutable,
quand on sait s'y prendre>, fit observer John Smith,
le philosophe de la bande.

Puis le cort\`ege, sans s'inqui\'eter de
r\'eveiller ou non les matelots du dernier quart,
encore endormis, remonta l'escalier et reparut sur le pont.

C'\'etait une r\'evolte en r\`egle.
Seul de tous les officiers du bord, Young, un des
midshipmen, avait fait cause commune avec les r\'e\-vol\-t\'es.

Quant aux hommes de l'\'equipage, les h\'esitants
avaient d\^u c\'eder pour l'instant, tandis que
les autres, sans armes, sans chef, restaient spectateurs
du drame qui allait s'accomplir sous leurs yeux.

Tous \'etaient sur le pont, rang\'es
en silence; ils observaient la contenance de leur capitaine,
qui, demi nu, s'avan\c{c}ait la t\^ete haute au milieu
de ces hommes habitu\'es \`a trembler devant lui. 

<Bligh, dit Christian, d'une voix rude, vous \^etes
d\'emont\'e de votre commandement.

--- Je ne vous reconnais pas le droit... r\'epondit le capitaine.

--- Ne perdons pas de temps en protestations inutiles,
s'\'ecria Christian, qui interrompit Bligh. Je suis,
en ce moment, l'interpr\`ete de tout l'\'equipage
de la {\sl Bounty}.
Nous n'avions pas encore quitt\'e l'Angleterre que nous
avions d\'ej\`a \`a nous plaindre de vos soup\c{c}ons
injurieux, de vos proc\'ed\'es brutaux. Lorsque je dis
nous, j'entends les officiers aussi bien que les matelots.
Non seulement nous n'avons jamais pu obtenir la satisfaction
qui nous \'etait due, mais vous avez toujours rejet\'e
nos plaintes avec m\'epris! Sommes-nous donc des chiens,
pour \^etre injuri\'es \`a tout moment?
Canailles, brigands, menteurs, voleurs! vous n'aviez
pas d'expression assez forte, d'injure assez grossi\`ere
pour nous! En v\'erit\'e, il faudrait ne pas \^etre un
homme pour supporter pareille existence!
Et moi, moi votre compatriote, moi qui connais votre
famille, moi qui ai d\'ej\`a fait deux voyages sous
vos ordres, m'avez-vous \'epargn\'e?
Ne m'avez-vous pas accus\'e, hier encore, de vous avoir
vol\'e quelques mis\'erables fruits? Et les hommes!
Pour un rien, aux fers!
Pour une bagatelle, vingt-quatre coups de corde!
Eh bien! tout se paie en ce monde!
Vous avez \'et\'e trop lib\'eral avec nous, Bligh!
A notre tour! Vos injures, vos injustices,
vos accusations insens\'ees, les tortures morales
et physiques dont vous avez accabl\'e votre \'equipage
depuis un an et demi, vous allez les expier,
et les expier durement!
Capitaine, vous avez \'et\'e jug\'e par ceux que vous avez
offens\'es, et vous \^etes condamn\'e.
Est-ce bien cela, camarades?

--- Oui, oui, \`a mort! s'\'ecri\`erent
la plupart des matelots, en mena\c{c}ant leur capitaine.

--- Capitaine Bligh, reprit Christian, quelques-uns
avaient parl\'e de vous hisser au bout d'une corde
entre le ciel et l'eau. D'autres proposaient de vous
d\'echirer les \'epaules avec le chat \`a neuf queues,
jusqu'\`a ce que mort s'ensuiv{\^\i}t. Ils manquaient
d'imagination. J'ai trouv\'e mieux que cela. D'ailleurs,
vous n'\^etes pas seul coupable ici. Ceux qui ont toujours
fid\`element ex\'ecut\'e vos ordres, si cruels qu'ils
fussent, seraient au d\'esespoir de passer sous mon
commandement. Ils ont m\'erit\'e de vous accompagner
l\`a o\`u le vent vous poussera.

--- Qu'on am\`ene la chaloupe!>

Un murmure d\'esapprobateur accueillit ces derni\`eres
paroles de Christian, qui ne parut pas s'en inqui\'eter.
Le capitaine Bligh, que ces menaces ne parvenaient pas
\`a troubler, profita d'un instant de silence pour prendre
la parole.

\pageinsert
\vskip 15cm
\centerline{\fig{img/bounty1.jpg}}
\smallskip
\centerline{\it <Offciers et matelots, dit-il d'une voix ferme...>}
\vfill
\endinsert

<Officiers et matelots, dit-il d'une voix ferme, en ma
qualit\'e d'officier de la marine royale, commandant
la {\sl Bounty}, je proteste contre le traitement que
vous voulez me faire subir. Si vous avez \`a vous plaindre
de la fa\c{c}on dont j'ai exerc\'e mon commandement,
vous pouvez me faire juger par une cour martiale.
Mais vous n'avez pas r\'efl\'echi, sans doute, \`a la
gravit\'e de l'acte que vous allez commettre. Porter
la main sur votre capitaine, c'est vous mettre
en r\'evolte contre les lois existantes, c'est rendre
pour vous tout retour impossible dans votre patrie,
c'est vouloir \^etre trait\'es comme des forbans!
T\^ot ou tard, c'est la mort ignominieuse, la mort
des tra{\^\i}tres et des rebelles! Au nom de l'honneur
et de l'ob\'eissance que vous m'avez jur\'es, je vous
somme de rentrer dans le devoir!

--- Nous savons parfaitement \`a quoi nous nous exposons,
r\'e\-pon\-dit Churchill.

--- Assez! Assez! cria l'\'equipage,
pr\^et \`a se porter \`a des voies de fait.

--- Eh bien! dit Bligh, s'il vous faut une victime, que ce
soit moi, mais moi seul! Ceux de mes compagnons que vous
condamnez comme moi, n'ont fait qu'ex\'ecuter mes ordres!>

La voix du capitaine fut alors couverte par un concert
de vocif\'erations, et il dut renoncer \`a toucher ces
c{\oe}urs devenus impitoyables.

Pendant ce temps, les dispositions avaient \'et\'e prises
pour que les ordres de Christian fussent ex\'ecut\'es.

Cependant, un assez vif d\'ebat s'\'etait \'elev\'e entre
le second et plusieurs des r\'evolt\'es qui voulaient
abandonner sur les flots le capitaine Bligh et ses compagnons
sans leur donner une arme, sans leur laisser une once de pain.

Quelques-uns -- et c'\'etait l'avis de Churchill -- trouvaient
que le nombre de ceux qui devaient quitter le navire n'\'etait
pas assez consid\'erable. Il fallait se d\'efaire, disait-il,
de tous les hommes qui, n'ayant pas tremp\'e directement dans
le complot, n'\'etaient pas s\^urs.
On ne pouvait compter sur ceux qui se contentaient d'accepter
les faits accomplis. Quant \`a lui, son dos lui faisait encore
mal des coups de fouet qu'il avait re\c{c}us pour avoir
d\'esert\'e \`a Tahiti. Le meilleur, le plus rapide moyen
de le gu\'erir, ce serait de lui livrer d'abord le commandant!...
Il saurait bien se venger, et de sa propre main!

<Hayward! Hallett! cria Christian, en s'adressant \`a deux
des officiers, sans tenir compte des observations de Churchill,
descendez dans la chaloupe.

--- Que vous ai-je fait, Christian, pour que vous me traitiez
ainsi? dit Hayward. C'est \`a la mort que vous m'envoyez!

--- Les r\'ecriminations sont inutiles!
Ob\'eissez, ou sinon!... Fryer, embarquez aussi!>

Mais ces officiers, au lieu de se diriger vers la chaloupe,
se rapproch\`erent du capitaine Bligh, et Fryer, qui semblait
le plus d\'etermin\'e, se pencha vers lui en disant:

<Commandant, voulez-vous essayer de reprendre le b\^atiment?
Nous n'avons pas d'armes, il est vrai; mais ces mutins,
surpris, ne pourront r\'esister. Si quelques-uns d'entre nous
sont tu\'es, qu'importe! On peut tenter la partie!
Que vous en semble?>

D\'ej\`a les officiers prenaient leurs dispositions
pour se jeter sur les r\'evolt\'es, occup\'es
\`a d\'epasser la chaloupe de ses porte-manteaux,
lorsque Churchill, \`a qui cet entretien, si rapide
qu'il f\^ut, n'avait pas \'echapp\'e,
les entoura avec quelques hommes bien arm\'es,
et les fit embarquer de force.

<Millward, Muspratt, Birket, et vous autres, dit Christian
en s'adressant \`a quelques-uns des matelots qui n'avaient
point pris part \`a la r\'evolte, descendez dans l'entrepont,
et choisissez ce que vous avez de plus pr\'ecieux!
Vous accompagnez le capitaine Bligh. Toi, Morrison,
surveille-moi ces gaillards-l\`a! Purcell, prenez votre
coffre de charpentier, je vous permets de l'emporter.>

Deux m\^ats avec leurs voiles, quelques clous, une scie,
une demi-pi\`ece de toile \`a voile, quatre petites pi\`eces
contenant cent vingt-cinq litres d'eau, cent cinquante livres
de biscuit, trente-deux livres de porc sal\'e, six bouteilles
de vin, six bouteilles de rhum, la cave \`a liqueur du capitaine,
voil\`a tout ce que les abandonn\'es eurent permission d'emporter.

On leur jeta, en outre, deux ou trois vieux sabres,
mais on leur refusa toute esp\`ece d'armes \`a feu.

<O\`u sont donc Heywood et Steward? dit Bligh, quand
il fut dans la chaloupe. Eux aussi m'ont-ils trahi?>

Ils ne l'avaient pas trahi, mais Christian avait r\'esolu
de les garder \`a bord.

Le capitaine eut alors un moment de d\'ecouragement
et de faiblesse bien pardonnable, qui ne dura pas.

<Christian, dit-il, je vous donne ma parole d'honneur
d'oublier tout ce qui vient de se passer, si vous renoncez
\`a votre abominable projet! Je vous en supplie, pensez
\`a ma femme et \`a ma famille!
Moi mort, que deviendront tous les miens!

--- Si vous aviez eu quelque honneur, r\'epondit Christian,
les choses n'en seraient point arriv\'ees \`a ce point.
Si vous-m\^eme aviez pens\'e un peu plus souvent \`a votre femme,
\`a votre famille, aux femmes et aux familles des autres, vous
n'auriez pas \'et\'e si dur, si injuste envers nous tous!>

A son tour, le bosseman, au moment d'embarquer,
essaya d'at\-ten\-drir Christian.
Ce fut en vain.

<Il y a longtemps que je souffre, r\'epondit ce dernier
avec amertume. Vous ne savez pas quelles ont \'et\'e mes
tortures! Non! cela ne pouvait durer un jour de plus, et,
d'ailleurs, vous n'ignorez pas que, durant tout le voyage,
moi, le second de ce navire, j'ai \'et\'e trait\'e comme
un chien! Cependant, en me s\'eparant du capitaine Bligh,
que je ne reverrai probablement jamais, je veux bien, par
piti\'e, ne pas lui enlever tout espoir de salut.
--|Smith! descendez dans la cabine du capitaine, et reportez-lui
ses v\^etements, sa commission, son journal et son portefeuille.
De plus, qu'on lui remette mes tables nautiques et mon propre
sextant. Il aura ainsi quelque chance de pouvoir sauver ses
compagnons et se tirer d'affaire lui-m\^eme!>

Les ordres de Christian furent ex\'ecut\'es,
non sans quelque protestation.

<Et maintenant, Morrison, largue l'amarre, cria
le second devenu le premier, et \`a la gr\^ace de Dieu!>

\pageinsert
\vskip 15cm
\centerline{\fig{img/bounty2.jpg}}
\smallskip
\centerline{\it <Tandis que les r\'evolt\'es
  saluaient d'acclamations ironiques...>}
\vfill
\endinsert

Tandis que les r\'evolt\'es saluaient d'acclamations
ironiques le capitaine Bligh et ses malheureux compagnons,
Christian, appuy\'e contre le bastingage, ne pouvait
d\'etacher les yeux de la chaloupe qui s'\'eloignait.
Ce brave officier, dont la conduite, jusqu'alors loyale
et franche, avait m\'erit\'e les \'eloges de tous les
commandants sous lesquels il avait servi, n'\'etait
plus aujourd'hui que le chef d'une bande de forbans.
Il ne lui serait plus permis de revoir ni sa vieille
m\`ere, ni sa fianc\'ee, ni les rivages de l'{\^\i}le
de Man, sa patrie. Il se sentait d\'echu dans sa propre
estime, d\'eshonor\'e aux yeux de tous!
Le ch\^atiment suivait d\'ej\`a la faute!

\vfill\eject


\topglue 5pc
\centerline{\sf II}
\bigskip
\centerline{\sf LES ABANDONN\'ES}
\vskip 2pc
\noindent
Avec ses dix-huit passagers, officiers et matelots,
et le peu de provisions qu'elle contenait, la chaloupe
qui portait Bligh \'etait tellement charg\'ee, qu'elle
d\'epassait \`a peine de quinze pouces le niveau de
la mer. Longue de vingt et un pieds, large de six, elle
pouvait \^etre parfaitement appropri\'ee au service de
la {\sl Bounty\/}; mais, pour contenir un \'equipage
aussi nombreux, pour faire un voyage un peu long, il
\'etait difficile de trouver embarcation plus d\'etestable.

Les matelots, confiants dans l'\'energie et l'habilet\'e
du capitaine Bligh et des officiers confondus dans le
m\^eme sort, nageaient vi\-gou\-reu\-se\-ment, et la chaloupe
fendait rapidement les lames.

Bligh n'avait pas h\'esit\'e sur le parti \`a prendre.
Il fallait, tout d'abord, regagner au plus t\^ot l'{\^\i}le
Tofoa, la plus voisine du groupe des {\^\i}les des Amis,
qu'ils avaient quitt\'ee quelques jours avant, il fallait
y recueillir des fruits de l'arbre \`a pain, renouveler
l'approvisionnement d'eau, et, de l\`a, courir sur Tonga-Tabou.
On pourrait sans doute y prendre des vivres en assez
grande quantit\'e pour faire la travers\'ee jusqu'aux
\'etablissements hollandais de Timor, si, par crainte
des indig\`enes, l'on ne voulait pas s'arr\^eter dans
les innombrables archipels sem\'es sur la route.

La premi\`ere journ\'ee se passa sans incident,
et la nuit tombait, lorsqu'on d\'ecouvrit les
c\^otes de Tofoa. Par malheur, le rivage \'etait
si rocheux, la plage si accore, qu'on ne pouvait y
d\'ebarquer de nuit. Il fallut donc attendre le jour.

Bligh, \`a moins de n\'ecessit\'e absolue,
entendait ne pas toucher aux provisions de la chaloupe.
Il fallait donc que l'{\^\i}le nourr{\^\i}t ses hommes
et lui. Cela semblait devoir \^etre difficile, car,
tout d'abord, lorsqu'ils furent \`a terre, ils ne
rencontr\`erent pas trace d'habitants.
Quelques-uns, cependant, ne tard\`erent pas \`a
se montrer, et, ayant \'et\'e bien re\c{c}us,
ils en amen\`erent d'autres, qui apport\`erent
un peu d'eau et quelques noix de coco.

L'embarras de Bligh \'etait grand.
Que dire \`a ces naturels qui avaient d\'ej\`a
trafiqu\'e avec la {\sl Bounty\/} pendant sa derni\`ere
rel\^ache? A tout prix, il importait de leur cacher
la v\'erit\'e, afin de ne pas d\'etruire
le prestige dont les \'etrangers avaient \'et\'e
entour\'es jusqu'alors dans ces {\^\i}les.

Dire qu'ils \'etaient envoy\'es aux provisions
par le b\^atiment rest\'e au large?
Impossible, puisque la {\sl Bounty\/} n'\'etait pas
visible, m\^eme du haut des collines! Dire que le navire
avait fait naufrage, et que les indig\`enes voyaient
en eux les seuls survivants des naufrag\'es? c'\'etait
encore la fable la plus vraisemblable. Peut-\^etre les
toucherait-elle, les am\`enerait-elle \`a
compl\'eter les provisions de la chaloupe.
Bligh s'arr\^eta donc \`a ce dernier parti,
si dangereux qu'il f\^ut, et il pr\'evint ses hommes,
afin que tout le monde f\^ut d'accord sur cette fable.

En entendant ce r\'ecit, les naturels ne firent
para{\^\i}tre ni marque de joie ni signes de chagrin.
Leur visage n'exprima qu'un profond \'etonnement,
et il fut impossible de conna{\^\i}tre ce qu'ils pensaient.

Le 2 mai, le nombre des indig\`enes venus des autres
parties de l'{\^\i}le s'accrut d'une fa\c{c}on
inqui\'etante, et Bligh put bient\^ot juger qu'ils
avaient des intentions hostiles. Quelques-uns
essay\`erent m\^eme de haler l'embarcation sur
le rivage, et ne se retir\`erent que devant les
d\'emonstrations \'energiques du capitaine, qui dut
les menacer de son coutelas. Pendant ce temps,
quelques-uns de ses hommes, que Bligh avait envoy\'es
\`a la recherche, rapportaient trois gallons d'eau.

Le moment \'etait venu de quitter cette {\^\i}le
inhospitali\`ere. Au coucher du soleil, tout \'etait
pr\^et, mais il n'\'etait pas facile de gagner
la chaloupe. Le rivage \'etait bord\'e d'une
foule d'indig\`enes qui choquaient des pierres l'une
contre l'autre, pr\^ets \`a les lancer.
Il fallait donc que la chaloupe se t{\^\i}nt \`a quelques
toises du rivage et n'accost\^at qu'au moment m\^eme o\`u
les hommes seraient tout \`a fait pr\^ets \`a embarquer.

Les Anglais, v\'eritablement tr\`es inquiets
des dispositions hostiles des naturels, redescendirent
la gr\`eve, au milieu de deux cents indig\`enes,
qui n'attendaient qu'un signal pour se jeter sur eux.
Cependant, tous venaient d'entrer heureusement dans
l'embarcation, lorsque l'un des matelots, nomm\'e
Bancroft, eut la funeste id\'ee de revenir sur
la plage pour chercher quelque objet qu'il y avait
oubli\'e. En une seconde, cet imprudent fut
entour\'e par les naturels et assomm\'e
\`a coups de pierre, sans que ses compagnons,
qui ne poss\'edaient pas une arme \`a feu,
pussent le secourir. D'ailleurs, eux-m\^emes,
\`a cet instant, \'etaient attaqu\'es,
des pierres pleuvaient sur eux.

<Allons, gar\c{c}ons, cria Bligh,
vite aux avirons, et souquez ferme!>

Les naturels entr\`erent alors dans la mer et firent
pleuvoir sur l'embarcation une nouvelle gr\^ele de
cailloux. Plusieurs hommes furent bless\'es. Mais Hayward,
ramassant une des pierres qui \'etaient tomb\'ees dans
la chaloupe, visa l'un des assaillants et l'atteignit
entre les deux yeux. L'indig\`ene tomba \`a la renverse
en poussant un grand cri auquel r\'epondirent les hourras
des Anglais. Leur infortun\'e camarade \'etait veng\'e.

Cependant, plusieurs pirogues se d\'etachaient du
rivage et leur donnaient la chasse. Cette poursuite
ne pouvait se terminer que par un combat, dont l'issue
n'aurait pas \'et\'e heureuse, lorsque le ma{\^\i}tre
d'\'equipage eut une lumineuse id\'ee.
Sans se douter qu'il imitait Hippom\`ene dans sa lutte
avec Atalante, il se d\'epouilla de sa vareuse et la jeta
\`a la mer. Les naturels, l\^achant la proie pour l'ombre,
s'attard\`erent afin de la ramasser, et cet exp\'edient
permit \`a la chaloupe de doubler la pointe de la baie.

Sur ces entrefaites, la nuit \'etait enti\`erement
venue, et les indig\`enes, d\'ecourag\'es,
abandonn\`erent la poursuite de la chaloupe.

Cette premi\`ere tentative de d\'ebarquement \'etait
trop mal\-heu\-reuse pour \^etre renouvel\'ee;
tel fut du moins l'avis du capitaine Bligh.

<C'est maintenant qu'il faudra prendre une r\'esolution,
dit-il. La sc\`ene qui vient de se passer \`a Tofoa
se renouvellera, j'en suis certain, \`a Tonga-Tabou,
et partout o\`u nous voudrons accoster. En petit nombre,
sans armes \`a feu, nous serons absolument \`a la
merci des indig\`enes. Priv\'es d'objets
d'\'echange, nous ne pouvons acheter de vivres,
et il nous est impossible de nous les procurer de vive force.
Nous sommes donc r\'eduits \`a nos seules
ressources. Or, vous savez comme moi, mes amis, combien
elles sont mis\'erables! Mais ne vaut-il pas mieux
s'en contenter que de risquer, \`a chaque atterrissage,
la vie de plusieurs d'entre nous? Cependant, je ne veux en
rien vous dissimuler l'horreur de notre situation.
Pour atteindre Timor, nous avons \`a peu pr\`es
douze cents lieues \`a franchir, et il faudra vous
contenter d'une once de biscuit par jour et d'un quart
de pinte d'eau! Le salut est \`a ce prix seulement,
et encore, \`a la condition que je trouverai en vous
la plus compl\`ete ob\'eissance.
R\'epondez-moi sans arri\`ere-pens\'ee!
Consentez-vous \`a tenter l'entreprise? 
Jurez-vous d'ob\'eir \`a mes ordres quels
qu'ils soient? Promettez-vous de vous soumettre
sans murmure \`a ces privations?

--- Oui, oui, nous le jurons! s'\'ecri\`erent
d'une commune voix les compagnons de Bligh.

--- Mes amis, reprit le capitaine, il faut aussi oublier
nos torts r\'eciproques, nos antipathies et nos haines,
sacrifier en un mot nos rancunes personnelles \`a
l'int\'er\^et de tous, qui doit seul nous guider!

--- Nous le promettons.

--- Si vous tenez votre parole, ajouta Bligh, et, au besoin,
je saurai vous y forcer, je r\'eponds du salut.>

La route fut faite alors vers l'O.-N.-O. Le vent,
qui \'etait assez fort, souilla en temp\^ete
dans la soir\'ee du 4 mai. Les lames devinrent
si grosses, que l'embarcation disparaissait entre elles,
et semblait ne pouvoir se relever. Le danger augmentait
\`a chaque instant. Tremp\'es et glac\'es,
les malheureux n'eurent pour se r\'econforter,
ce jour-l\`a, qu'une tasse \`a th\'e de rhum et
le quart d'un fruit \`a pain \`a moiti\'e pourri.

\pageinsert
\vskip 15cm
\centerline{\fig{img/bounty3.jpg}}
\smallskip
\centerline{\it <Les lames devinrent si grosses...>}
\vfill
\endinsert

Le lendemain et les jours suivants, la situation
ne changea pas. L'embarcation passa au milieu
d'{\^\i}les innombrables, d'o\`u quelques
pirogues se d\'etach\`erent.

Etait-ce pour lui donner la chasse, \'etait-ce
pour faire quelques \'echanges? Dans le doute,
il aurait \'et\'e imprudent de s'arr\^eter.
Aussi, la chaloupe, les voiles gonfl\'ees par un bon vent,
les eut bient\^ot laiss\'ees loin derri\`ere elle.

Le 9 mai, un orage \'epouvantable \'eclata.
Le tonnerre, les \'eclairs se succ\'edaient
sans interruption. La pluie tombait avec une force
dont les plus violents orages de nos climats ne peuvent
donner une id\'ee. Il \'etait impossible
de faire s\'echer les v\^etements.
Bligh, alors, eut l'id\'ee de les tremper dans
l'eau de mer et de les impr\'egner de sel, afin
de ramener \`a la peau un peu de la chaleur
enlev\'ee par la pluie. Toutefois, ces pluies
torrentielles, qui caus\`erent tant de souffrances
au capitaine et \`a ses compagnons, leur
\'epargn\`erent d'autres tortures encore
plus horribles, les tortures de la soif, qu'une insoutenable
chaleur e\^ut bient\^ot provoqu\'ees.

Le 17 mai, au matin, \`a la suite d'un orage terrible,
les plaintes devinrent unanimes:

<Jamais nous n'aurons la force d'atteindre la Nouvelle-Hollande,
s'\'ecri\`erent les malheureux. Transperc\'es
par la pluie, \'epuis\'es de fatigue, n'aurons-nous
jamais un moment de repos! A demi morts de faim,
n'augmenterez-vous pas nos rations, capitaine? Peu importe
que nos vivres s'\'epuisent! Nous trouverons facilement
\`a les remplacer en arrivant \`a la Nouvelle-Hollande!

--- Je refuse, r\'epondit Bligh.
Ce serait agir comme des fous.
Comment! nous n'avons franchi que la moiti\'e
de la distance qui nous s\'epare de l'Australie,
et vous \^etes d\'ej\`a d\'ecourag\'es!
Croyez-vous, d'ailleurs, pouvoir trouver facilement des
vivres sur la c\^ote de la Nouvelle-Hollande?
Vous ne connaissez donc pas le pays et ses habitants.>

Et Bligh se mit \`a peindre \`a grands traits
la nature du sol, les m{\oe}urs des indig\`enes,
le peu de fonds qu'il fallait faire sur leur accueil,
toutes choses que son voyage avec le capitaine Cook
lui avait appris \`a conna{\^\i}tre.
Pour cette fois encore, ses infortun\'es compagnons
l'\'ecout\`erent et se turent.

Les quinze jours suivants furent \'egay\'es
par un clair soleil, qui permit de s\'echer les
v\^etements. Le 27 furent franchis les brisants
qui bordent la c\^ote orientale de la Nouvelle-Hollande.
La mer \'etait calme derri\`ere cette ceinture
madr\'eporique, et quelques groupes d'{\^\i}les,
\`a la v\'eg\'etation exotique,
r\'ejouissaient les regards.

On d\'ebarqua en ne s'avan\c{c}ant qu'avec
pr\'ecaution. On ne trouva d'autres traces du
s\'ejour des naturels que d'anciennes places
\`a feu. Il \'etait donc possible de passer
une bonne nuit \`a terre.
Mais il fallait manger. Par bonheur, un des matelots
d\'ecouvrit un banc d'hu{\^\i}tres.
Ce fut un v\'eritable r\'egal.

Le lendemain, Bligh trouva dans la chaloupe un verre
grossissant, un briquet et du soufre. Il fut donc \`a
m\^eme de se procurer du feu pour faire cuire le gibier
ou le poisson.

Bligh eut alors la pens\'ee de diviser son
\'equipage en trois escouades: l'une devait tout
mettre en ordre dans l'embarcation; les deux autres,
aller \`a la recherche des vivres. Mais plusieurs
hommes se plaignirent avec amertume, d\'eclarant
qu'ils aimaient mieux se passer de d{\^\i}ner que de
s'aventurer dans le pays.

L'un d'eux, plus violent ou plus \'enerv\'e
que ses camarades, alla m\^eme jusqu'\`a dire
au capitaine:

<Un homme en vaut un autre, et je ne vois pas pourquoi
vous resteriez toujours \`a vous reposer!
Si vous avez faim, allez chercher de quoi manger!
Pour ce que vous faites ici, je vous remplacerai bien!>

Bligh, comprenant que cet esprit de mutinerie devait
\^etre enray\'e sur-le-champ, saisit un coutelas,
et, en jetant un autre aux pieds du rebelle, il lui cria:

<D\'efends-toi, ou je te tue comme un chien!>

Cette attitude \'energique fit aussit\^ot rentrer
le mutin en lui-m\^eme, et le m\'econtentement
g\'en\'eral se calma.

Pendant cette rel\^ache, l'\'equipage de la chaloupe
r\'ecolta abondamment des hu{\^\i}tres, des peignes%
\fn{$^1$}{Esp\`ece de coquillage.} et de l'eau douce.

Un peu plus loin, dans le d\'etroit de l'Endeavour,
de deux d\'e\-ta\-che\-ments envoy\'es \`a la chasse
des tortues et des noddis\fn{$^2$}{Sorte d'oiseaux
({\it Anous stolidus\/}).}, le premier revint les mains vides;
le second rapporta six noddis, mais il en aurait pris bien
davantage sans l'obstination de l'un des chasseurs, qui,
s'\'etant \'ecart\'e de ses camarades,
effraya ces oiseaux. Cet homme avoua, plus tard,
qu'il s'\'etait empar\'e de neuf de ces
volatiles et qu'il les avait mang\'es crus sur place.

Sans les vivres et l'eau douce qu'il venait de trouver sur
la c\^ote de la Nouvelle-Hollande, il est bien certain que
Bligh et ses compagnons auraient p\'eri. D'ailleurs, tous
\'etaient dans un \'etat lamentable, h\^aves, d\'efaits,
\'epuis\'es|--|de v\'eritables cadavres.

Le voyage en pleine mer, pour gagner Timor, ne fut que
la douloureuse r\'ep\'etition des souffrances d\'ej\`a
endur\'ees par ces malheureux avant d'atteindre les
c\^otes de la Nouvelle-Hollande.
Seulement, la force de r\'esistance avait diminu\'e
chez tous, sans exception. Au bout de quelques jours,
leurs jambes \'etaient enfl\'ees. Dans cet \'etat de
faiblesse extr\^eme, ils \'etaient accabl\'es par une
envie de dormir presque continuelle. C'\'etaient
les signes avant-coureurs d'une fin qui ne pouvait
tarder beaucoup. Aussi Bligh, qui s'en aper\c{c}ut,
distribua une double ration aux plus affaiblis
et s'effor\c{c}a de leur rendre un peu d'espoir.

Enfin, le 12 juin au matin, la c\^ote de Timor apparut,
apr\`es trois mille six cent dix-huit lieues d'une
travers\'ee accomplie dans des conditions \'epouvantables.

L'accueil que les Anglais re\c{c}urent \`a Coupang
fut des plus sympathiques. Ils y rest\`erent deux mois
pour se refaire. Puis, Bligh, ayant achet\'e un petit schooner,
gagna Batavia, o\`u il s'embarqua pour l'Angleterre.

Ce fut le 14 mars 1790 que les abandonn\'es d\'ebarqu\`erent
\`a Ports\-mouth. Le r\'ecit des tortures qu'ils avaient
endur\'ees excita la sympathie universelle et l'indignation
de tous les gens de c{\oe}ur. Presque aussit\^ot, l'Amiraut\'e
proc\'edait \`a l'armement de la fr\'egate {\sl La Pandore},
de vingt-quatre canons et de cent soixante hommes
d'\'equipage, et l'envoyait \`a la poursuite
des r\'evolt\'es de la {\sl Bounty}.
On va voir ce qu'ils \'etaient devenus.

\vfill\eject


\topglue 5pc
\centerline{\sf III}
\bigskip
\centerline{\sf LES R\'EVOLT\'ES}
\vskip 2pc
\noindent
Apr\`es que le capitaine Bligh eut \'et\'e abandonn\'e en
pleine mer, la {\sl Bounty\/} avait fait voile pour Tahiti.
Le jour m\^eme, elle atteignait Touboua\"\i. Le riant aspect
de cette petite {\^\i}le, entour\'ee d'une ceinture
de roches madr\'eporiques, invitait Christian \`a y descendre;
mais les d\'emonstrations des habitants parurent trop
mena\c{c}antes, et le d\'ebarquement ne fut pas effectu\'e.

Ce fut le 6 juin 1789 que l'ancre tomba dans la rade
de Matava\"\i. La surprise des Tahitiens fut extr\^eme
en reconnaissant la {\sl Bounty}. Les r\'evolt\'es
retrouv\`erent l\`a les indig\`enes avec lesquels ils
avaient \'et\'e en rapport dans une pr\'ec\'edente
rel\^ache, et ils leur racont\`erent une fable,
\`a laquelle ils eurent soin de m\^eler le nom du
capitaine Cook, dont les Tahitiens avaient conserv\'e
le meilleur souvenir.

Le 29~juin, les r\'evolt\'es repartirent pour
Touboua{\"\i} et se mirent en qu\^ete de quelque
{\^\i}le qui f\^ut situ\'ee en dehors de la route
ordinaire des b\^atiments, dont le sol f\^ut assez
fertile pour les nourrir, et sur laquelle ils pussent vivre
en toute s\'ecurit\'e.
Ils err\`erent ainsi d'archipel en archipel,
commettant toutes sortes de d\'epr\'edations
et d'exc\`es, que l'autorit\'e de Christian
ne parvenait que bien rarement \`a pr\'evenir.

Puis, attir\'es encore une fois par la fertilit\'e
de Tahiti, par les m{\oe}urs douces et faciles de ses
habitants, ils regagn\`erent la baie de Matava\"\i.
L\`a, les deux tiers de l'\'equipage descendirent
aussit\^ot \`a terre. Mais, le soir m\^eme,
la {\sl Bounty\/} avait lev\'e l'ancre et disparu,
avant que les matelots d\'ebarqu\'es eussent pu
soup\c{c}onner l'intention de Christian de partir sans eux.

\pageinsert
\vskip 15cm
\centerline{\fig{img/bounty4.jpg}}
\smallskip
\centerline{\it <L'ancre tomba dans la rade de Matava\"\i.>}
\vfill
\endinsert

Livr\'es \`a eux-m\^emes, ces hommes s'\'etablirent
sans trop de regrets dans diff\'erents districts de
l'{\^\i}le. Le ma{\^\i}tre d'\'equipage Stewart et
le midshipman Peter Heywood, les deux officiers que
Christian avait except\'es de la condamnation
prononc\'ee contre Bligh, et avait emmen\'es
malgr\'e eux, rest\`erent \`a Matava{\"\i}
aupr\`es du roi Tippao, dont Stewart \'epousa
bient\^ot la s{\oe}ur. Morrison et Millward se rendirent
aupr\`es du chef P\'eno, qui leur fit bon accueil.
Quant aux autres matelots, ils s'enfonc\`erent dans
l'int\'erieur de l'{\^\i}le et ne tard\`erent
pas \`a \'epouser des Tahitiennes.

Churchill et un fou furieux nomm\'e Thompson, apr\`es
avoir commis toute sorte de crimes, en vinrent tous
deux aux mains. Churchill fut tu\'e dans cette lutte,
et Thompson lapid\'e par les naturels. Ainsi p\'erirent
deux des r\'evolt\'es qui avaient pris la plus grande
part \`a la r\'ebellion. Les autres surent, au contraire,
par leur bonne conduite, se faire ch\'erir des Tahitiens.

Cependant, Morrison et Millward voyaient toujours
le ch\^atiment suspendu sur leurs t\^etes et
ne pouvaient vivre tranquilles dans cette {\^\i}le
o\`u ils auraient \'et\'e facilement d\'ecouverts.
Ils con\c{c}urent donc le dessein de construire un schooner,
sur lequel ils essayeraient de gagner Batavia,
afin de se perdre au milieu du monde civilis\'e.
Avec huit de leurs compagnons, sans autres outils que ceux
du charpentier, ils parvinrent, non sans peine,
\`a construire un petit b\^atiment qu'ils
appel\`erent la {\sl R\'esolution},
et ils l'amarr\`erent dans une baie derri\`ere
une des pointes de Tahiti, nomm\'ee la pointe V\'enus.
Mais l'impossibilit\'e absolue o\`u ils se trouvaient
de se procurer des voiles les emp\^echa de prendre la mer.

Pendant ce temps, forts de leur innocence, Stewart
cultivait un jardin, et Peter Heywood r\'eunissait
les mat\'eriaux d'un vocabulaire, qui fut, plus tard,
d'un grand secours aux missionnaires anglais.

Cependant, dix-huit mois s'\'etaient \'ecoul\'es
lorsque, le 23~mars 1791, un vaisseau doubla la pointe
V\'enus et s'arr\^eta dans la baie Matava\"\i.
C'\'etait la {\sl Pandore}, envoy\'ee \`a la poursuite
des r\'evolt\'es par l'Amiraut\'e anglaise.

Heywood et Stewart s'empress\`erent de se rendre
\`a bord, d\'e\-cla\-r\`erent leurs noms
et qualit\'es, racont\`erent qu'ils n'avaient
pris aucune part \`a la r\'evolte; mais on ne les
crut pas, et ils furent aussit\^ot mis aux fers,
ainsi que tous leurs compagnons, sans que la moindre
enqu\^ete e\^ut \'et\'e faite.
Trait\'es avec l'inhumanit\'e la plus
r\'evoltante, charg\'es de cha{\^\i}nes,
menac\'es d'\^etre fusill\'es s'ils se
servaient de la langue tahitienne pour converser entre eux,
ils furent enferm\'es dans une cage de onze pieds
de long, plac\'ee \`a l'extr\'emit\'e
du gaillard d'arri\`ere, et qu'un amateur de mythologie
d\'ecora du nom de <bo{\^\i}te de Pandore>.

Le 19~mai, la {\sl R\'esolution}, qui avait
\'et\'e pourvue de voiles, et la {\sl Pandore\/}
reprirent la mer. Pendant trois mois, ces deux b\^atiments
crois\`erent \`a travers l'archipel des Amis,
o\`u l'on supposait que Christian et le reste des
r\'evolt\'es avaient pu se r\'efugier.
La {\sl R\'esolution}, d'un faible tirant d'eau,
rendit m\^eme de grands services pendant cette
croisi\`ere; mais elle disparut dans les parages
de l'{\^\i}le Chatam, et, bien que la {\sl Pandore\/}
f\^ut rest\'ee plusieurs jours en vue, jamais
on n'en entendit parler, ni des cinq marins qui la montaient.

La {\sl Pandore\/} avait repris la route d'Europe avec
ses prisonniers, lorsque, dans le d\'etroit de Torr\`es,
elle donna contre un \'ecueil de corail et sombra
presque aussit\^ot avec trente et un de ses matelots
et quatre des r\'evolt\'es.

L'\'equipage et les prisonniers, qui avaient
\'echapp\'e au naufrage, gagn\`erent
alors un {\^\i}lot sablonneux.
L\`a, les officiers et les matelots purent
s'abriter sous des tentes ; mais les rebelles,
expos\'es aux ardeurs d'un soleil vertical,
furent r\'eduits, pour trouver un peu de
soulagement, \`a s'enfoncer dans le sable jusqu'au cou.

Les naufrag\'es rest\`erent sur cet {\^\i}lot
pendant quelques jours; puis, tous gagn\`erent Timor
dans les chaloupes de la {\sl Pandore}, et la surveillance
si rigoureuse dont les mutins \'etaient l'objet
ne fut pas un moment n\'eglig\'ee,
malgr\'e la gravit\'e des circonstances.

Arriv\'es en Angleterre au mois de juin 1792,
les r\'evolt\'es pass\`erent devant un
conseil de guerre pr\'esid\'e par l'amiral Hood.
Les d\'ebats dur\`erent six jours et se
termin\`erent par l'acquittement de quatre des
accus\'es et la condamnation \`a mort des six autres,
pour crime de d\'esertion et enl\`evement
du b\^atiment confi\'e \`a leur garde.
Quatre des condamn\'es furent pendus \`a bord
d'un vaisseau de guerre; les deux autres, Stewart
et Peter Heywood, dont l'innocence avait enfin \'et\'e
reconnue, furent graci\'es.

Mais qu'\'etait devenue la {\sl Bounty\/}?
Avait-elle fait naufrage avec les derniers des
r\'evolt\'es? Voil\`a ce qu'il
\'etait impossible de savoir.

En 1814, vingt-cinq ans apr\`es la sc\`ene
par laquelle ce r\'ecit commence, deux navires de guerre
anglais croisaient en Oc\'eanie sous le commandement
du capitaine Staines.
Ils se trouvaient, au sud de l'archipel Dangereux,
en vue d'une {\^\i}le montagneuse et volcanique
que Carteret avait d\'ecouverte dans son voyage
autour du monde, et \`a laquelle il avait donn\'e
le nom de Pitcairn.
Ce n'\'etait qu'un c\^one, presque sans rivage,
qui s'\'elevait \`a pic au-dessus de la mer,
et que tapissaient jusqu'\`a sa cime des for\^ets
de palmiers et d'arbres \`a pain.
Jamais cette {\^\i}le n'avait \'et\'e visit\'ee;
elle se trouvait \`a douze cents milles de Tahiti, par
$25^\circ\,4'$ de latitude sud et $180^\circ\,8'$%
\fn{$^3$}{Faute d'impression? Longitude correcte: $133^\circ\,6'$}
de longitude ouest; elle ne mesurait que quatre milles
et demi \`a sa circonf\'erence, et un mille et demi
seulement \`a son grand axe, et l'on n'en savait que
ce qu'en avait rapport\'e Carteret.

Le capitaine Staines r\'esolut de la reconna{\^\i}tre
et d'y chercher un endroit convenable pour d\'ebarquer.

En s'approchant de la c\^ote, il fut surpris d'apercevoir
des cases, des plantations, et, sur la plage, deux naturels
qui, apr\`es avoir lanc\'e une embarcation \`a
la mer et travers\'e habilement le ressac,
se dirig\`erent vers son b\^atiment.
Mais son \'etonnement n'eut plus de bornes lorsqu'il
s'entendit interpeller, en excellent anglais, par cette phrase:

<H\'e! vous autres, allez-vous nous jeter une corde,
que nous montions \`a bord!>

\`A peine arriv\'es sur le pont, les deux robustes
rameurs furent entour\'es par les matelots stup\'efaits,
qui les accablaient de questions auxquelles ils ne savaient
que r\'epondre. Conduits devant le commandant,
ils furent interrog\'es r\'eguli\`erement.

<Qui \^etes-vous?

--- Je m'appelle Fletcher Christian, et mon camarade, Young.>

Ces noms ne disaient rien au capitaine Staines, qui \'etait
bien loin de penser aux survivants de la {\sl Bounty}.

<Depuis quand \^etes-vous ici?

--- Nous y sommes n\'es.

--- Quel \^age avez-vous?

--- J'ai vingt-cinq ans, r\'epondit Christian, et Young dix-huit.

--- Vos parents ont-ils \'et\'e jet\'es
sur cette {\^\i}le par quelque naufrage?>

Christian fit alors au capitaine Staines l'\'emouvante
confession qui va suivre et dont voici les principaux traits:

En quittant Tahiti, o\`u il abandonnait vingt et un
de ses camarades, Christian, qui avait \`a bord de
la {\sl Bounty\/} le r\'ecit de voyage du capitaine Carteret,
s'\'etait dirig\'e directement vers l'{\^\i}le Pitcairn,
dont la position lui avait sembl\'e convenir au but qu'il se
proposait. Vingt-huit hommes composaient encore l'\'equipage
de la {\sl Bounty\/}. C'\'etaient Christian, l'aspirant Young
et sept matelots, six Tahitiens pris \`a Tahiti, dont trois
avec leurs femmes et un enfant de dix mois, plus trois hommes
et six femmes, indig\`enes de Touboua\"\i. % Roubouaï?

Le premier soin de Christian et de ses compagnons, d\`es
qu'ils eurent atteint l'{\^\i}le Pitcairn, avait \'et\'e
de d\'etruire la {\sl Bounty}, afin de n'\^etre pas
d\'ecouverts. Sans doute, ils s'\'etaient enlev\'e
par l\`a toute possibilit\'e de quitter l'{\^\i}le,
mais le soin de leur s\'ecurit\'e l'exigeait.

L'\'etablissement de la petite colonie ne devait pas
se faire sans difficult\'es, avec des gens qu'unissait
seule la solidarit\'e d'un crime. De sanglantes querelles
avaient \'eclat\'e bient\^ot entre les Tahitiens
et les Anglais. Aussi, en 1794, quatre des mutins survivaient-ils
seulement. Christian \'etait tomb\'e sous le couteau
de l'un des indig\`enes qu'il avait amen\'es.
Tous les Tahitiens avaient \'et\'e massacr\'es.

Un des Anglais, qui avait trouv\'e le moyen de fabriquer
des spi\-ri\-tueux avec la racine d'une plante indig\`ene,
avait fini par s'abrutir dans l'ivresse, et, pris d'un acc\`es
de {\sl delirium tremens}, s'\'etait pr\'ecipit\'e
du haut d'une falaise dans la mer.

Un autre, en proie \`a un acc\`es de folie furieuse,
s'\'etait jet\'e sur Young et sur un des matelots, nomm\'e
John Adams, qui s'\'etaient vus forc\'es de le tuer. En
1800, Young \'etait mort pendant une violente crise d'asthme.

John Adams fut alors le dernier survivant de l'\'equipage
des r\'evolt\'es.

Rest\'e seul avec plusieurs femmes et vingt enfants,
n\'es du mariage de ses camarades avec les Tahitiennes,
le caract\`ere de John Adams s'\'etait modifi\'e
profond\'ement. Il n'avait que trente-six ans alors;
mais, depuis bien des ann\'ees, il avait assist\'e
\`a tant de sc\`enes de violence et de carnage,
il avait vu la nature humaine sous de si tristes aspects,
qu'apr\`es avoir fait un retour sur lui-m\^eme,
il s'\'etait tout \`a fait amend\'e.

Dans la biblioth\`eque de la {\sl Bounty},
conserv\'ee sur l'{\^\i}le, se trouvaient une bible
et plusieurs livres de pri\`eres. John Adams, qui les
lisait fr\'equemment, se convertit, \'eleva dans
d'excellents principes la jeune population qui le
consid\'erait comme un p\`ere, et devint,
par la force des choses, le l\'egislateur, le grand
pr\^etre et, pour ainsi dire, le roi de Pitcairn.

Cependant, jusqu'en 1814, ses alarmes avaient \'et\'e
continuelles. En 1795, un b\^atiment s'\'etant
approch\'e de Pitcairn, les quatre survivants de la
{\sl Bounty\/} s'\'etaient cach\'es dans des bois
inaccessibles et n'avaient os\'e redescendre dans la baie
qu'apr\`es le d\'epart du navire. M\^eme acte
de prudence, lorsqu'en 1808, un capitaine am\'ericain%
\fn{$^4$}{Mayhew Folger}
d\'ebarqua sur l'{\^\i}le, o\`u il s'empara d'un
chronom\`etre et d'une boussole, qu'il fit parvenir
\`a l'Amiraut\'e anglaise; mais l'Amiraut\'e
ne s'\'emut pas \`a la vue de ces reliques de la
{\sl Bounty}. Il est vrai qu'elle avait en Europe des
pr\'eoccupations d'une bien autre gravit\'e,
\`a cette \'epoque.\fn{$^5$}{La guerre d'ind\'ependence
espagnole}

\pageinsert
\vskip 15cm
\centerline{\fig{img/bounty5.jpg}}
\smallskip
\centerline{\it <John Adams \'etait le seul survivant.>}
\vfill
\endinsert

Tel fut le r\'ecit fait au commandant Staines par les
deux naturels, Anglais par leurs p\`eres, l'un fils
de Christian, l'autre fils d'Young; mais, lorsque Staines
demanda \`a voir John Adams, celui-ci refusa de se
rendre \`a bord avant de savoir ce qu'il adviendrait de lui.

Le commandant, apr\`es avoir assur\'e aux deux
jeunes gens que John Adams \'etait couvert par la
prescription, puisque vingt-cinq ans s'\'etaient
\'ecoul\'es depuis la r\'evolte de la {\sl Bounty},
descendit \`a terre, et il fut re\c{c}u \`a son
d\'ebarquement par une population compos\'ee de
quarante-six adultes et d'un grand nombre d'enfants.
Tous \'etaient grands et vigoureux, avec le type anglais
nettement accus\'e; les jeunes filles surtout
\'etaient admirablement belles, et leur modestie leur
imprimait un caract\`ere tout \`a fait s\'eduisant.

Les lois mises en vigueur dans l'{\^\i}le \'etaient
des plus simples. Sur un registre \'etait not\'e
ce que chacun avait gagn\'e par son travail.
La monnaie \'etait inconnue; toutes les transactions
se faisaient au moyen de l'\'echange, mais il n'y avait
pas d'industrie, car les mati\`eres premi\`eres
manquaient. Les habitants portaient pour tout habillement
de vastes chapeaux et des ceintures d'herbe.
La p\^eche et l'agriculture, telles \'etaient
leurs principales occupations. Les mariages ne se faisaient
qu'avec la permission d'Adams, et lorsque l'homme avait
d\'efrich\'e et plant\'e un terrain assez
vaste pour subvenir \`a l'entretien de sa future famille.

Le commandant Staines, apr\`es avoir recueilli les documents
les plus curieux sur cette {\^\i}le, perdue dans les parages
les moins fr\'equent\'es du Pacifique, reprit la mer
et revint en Europe.

Depuis cette \'epoque, le v\'en\'erable John Adams
a termin\'e sa carri\`ere si accident\'ee.
Il est mort en 1829, et a \'et\'e remplac\'e
par le r\'ev\'erend George Nobbs, qui remplit encore
dans l'{\^\i}le les fonctions de pasteur, de m\'edecin
et de ma{\^\i}tre d'\'ecole.

En 1853, les descendants des r\'evolt\'es de la
{\sl Bounty\/} \'etaient au nombre de cent soixante-dix
individus. Depuis lors, la population ne fit que
s'accro{\^\i}tre, et devint si nombreuse, que, trois ans
plus tard, elle dut s'\'etablir en grande partie
sur l'{\^\i}le Norfolk, qui avait jusqu'alors servi
de station pour les convicts.
Mais une partie des \'emigr\'es regrettaient
Pitcairn, bien que Norfolk f\^ut quatre fois plus grande,
que son sol f\^ut remarquable par sa richesse, et que
les conditions de l'existence y fussent bien plus faciles.
Au bout de deux ans de s\'ejour, plusieurs familles
retourn\`erent \`a Pitcairn, o\`u elles
continuent \`a prosp\'erer.

Tel fut donc le d\'enouement d'une aventure
qui avait commenc\'e d'une fa\c{c}on si
tragique. Au d\'ebut, des r\'evolt\'es,
des assassins, des fous, et maintenant, sous l'influence
des principes de la morale chr\'etienne et de
l'instruction donn\'ee par un pauvre matelot
converti, l'{\^\i}le Pitcairn est devenue la patrie
d'une population douce, hospitali\`ere, heureuse,
chez laquelle se retrouvent les m{\oe}urs patriarcales
des premiers \^ages.

\nobreak\vskip 1pc
\centerline{\sc FIN}
\vfill\eject

\endfrench

% p30 (annexe: carte)

\topglue0pc
\centerline{\sf Annexe $\cdot$ Anhang}
\vfill
\hbox to 9.8cm{\hss
 \special{pdf:image width 15.3cm rotate 90 (map/BountyMap.pdf)}}
\eject

% p31 (annexe: toponymes)

\begingroup
\eightpoint\parindent=0pt\parskip=3pt plus 1pt minus 1pt
\def\:#1:{\par\noindent{\it#1\/}:\enspace\ignorespaces\hangindent1em}%
\leftline{\bf Ortsnamen \rm in der ungef\"ahren Reihenfolge
 ihrer Nennung in der Erz\"ahlung.}
\smallskip
\:Spithead: ein Teil des Solents, der Meerenge zwischen England
 und der Isle of Wight, vor Portsmouth; diente der Royal Navy
 als Liegeplatz f\"ur ihre Schiffe.
\:Portsmouth: damals wichtigster St\"utzpunkt der Royal Navy.
\:Antilles: Inselgruppe der Karibik, grenzt das Karibische Meer
 vom Atlantik ab.
\:Indes occidentales: die Antillen, welche von Kolumbus irrt\"umlich als
 Indien erachtet wurden, sp\"ater auch bewusst als Abgrenzung von Ostindien.
\:Tahiti: Hauptinsel der Gesellschaftsinseln (archipel de la
 Soci\'et\'e), weit im S\"udpazifik.
 Hier wurden die Brotfruchtb\"aume ({\sl Artocarpus altilis\/}) geholt.
\:Matava{\"\i} (la baie de): Bucht im Norden Tahitis. Hier erster
 Kontakt zwischen Polynesiern und Europ\"aern (James Cook 1768).
\:pointe V\'enus: Kap am \"ostlichen Ende der Matavai-Bucht.
 James Cook hat hier den Venus-Transit (Venus zwischen Sonne und
 Erde) von 1769 beobachtet.
\:{\^\i}les des Amis: der Archipel von Tonga, fr\"uher
 Freundschaftsinseln genannt weil James Cook 1773 freundlich
 empfangen wurde, heute das K\"onigreich Tonga.
\:Tofoa: Tafua, eine Vulkaninsel etwa 150\,km n\"ordlich
 von Tongatapu.
\:Tonga-Tabou: Tongatapu, die Hauptinsel des heutigen
 K\"onigreichs Tonga.
\:Timor: die heutige Insel Timor, n\"ordlich von Australien.
\:Nouvelle-Hollande: Australien vor der Besitznahme durch
 die Briten.
\:d\'etroit de l'Endeavour: s\"udlicher Teil der Torresstrasse.
\:d\'etroit de Torr\`es: Torresstrasse, Meerenge zwischen
 Australien und Neuguinea, 150\,km breit, geringe Tiefe,
 viele Inseln, Sandb\"anke, Korallenriffe, Klippen.
\:Coupang: die Stadt Kupang auf Timor, damals unter
 holl\"andischer Verwaltung.
\:Batavia: das heutige Jakarta, damals das
 Hauptquartier der Niederl\"andischen Ostindien-Kompanie.
 Batavia ist der lateinische Name f\"ur die Niederlande.
\:Touboua\"\i: Tubuai, kleine Vulkaninsel umgeben von einem
 Korallenriff, 650\,km s\"udlich von Tahiti,
 heute zu franz\"osisch Polynesien geh\"orend (wie auch Tahiti).
\:{\^\i}le Chatam: unklar, sicher {\it nicht\/} die Chathaminseln
 \"ostlich von Neuseeland.
\:archipel Dangereux: die Tuamotu-Inseln im S\"udpazifik;
 wegen der vielen Riffe fr\"uher auch Gef\"ahrliche Inseln genannt.
\:Pitcairn: einsame Insel im S\"udpazifik, 1767 von Philipp
 Carteret entdeckt und nach einem Seekadetten benannt,
 jedoch mit ungenauen Koordinaten ver\-merkt; Bligh und Christian
 wussten wohl davon und Christian hat die Insel wiedergefunden.
 Seit 1838 britisches \"Uberseegebiet und immer noch bewohnt.
\:Norfolk: Insel zwischen Australien und Neuseeland;
 zeitweise britische Straf\-ko\-lo\-nie;
 ab 1856 neue Heimat f\"ur einen Teil der Nachkommen der Bounty-Meuterer.
\medbreak
{\bf Masseinheiten.}
{\it Lieue\/}: eigentlich Leuge, Wegstunde, etwa 4\,km; hier eher die
 nautische Meile womit die ``3618 lieues'' etwa 6500\,km entsprechen.  
{\it Once\/}: Unze, etwa 30 Gramm.
{\it Pinte\/}: Pinte, nicht einheitlich,
 hier wohl die Pariser Pinte (etwa 1~Liter).
{\it Toise\/}: altes L\"angenmass, 6~Fuss oder knapp 2~Meter.
\medbreak
{\bf Glossar.}
{\it Vergue d'artimon\/}: Besanrah (Segelstange am hintersten Mast).
{\it Roches madr\'eporiques\/}: Korallenfelsen.
{\it Gaillard d'arri\`ere\/}: Achterdeck.

\endgroup
\vfill\eject

% p32 (multiple of four)

\topglue5pc
\vfill\nopagenumbers
\centerline{\eightpoint Verne: Les R\'evolt\'es de la Bounty}
\centerline{\eightpoint \copytext}
\supereject % also eject pending insertions, if any

\end

