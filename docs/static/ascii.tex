%&Plain
% ujr/2000-10-06 started
% ujr/2002-09-07 last change
% ujr/2022-11-01 add C escapes

\magnification=\magstep1
\nopagenumbers
\parindent=0pt

\def\horiz{\noalign{\hrule}}
\def\Vrule{\vrule width1pt}
\def\Strut{\lower1.2ex\null\raise2.5ex\null}

\font\titlefont=cmbx12 scaled \magstep1
\font\smallfont=cmti8

\centerline{\titlefont The ASCII Scheme of Character Encoding}
\bigskip%\medskip
ASCII is short for {\sl American Standard Code for Information
Interchange\/}, a standard mapping characters to 7 bit codes, that
is, to numbers ranging from 0 to 127~decimal, or {\it7F\/}~hexadecimal,
according to the following table. Code {\it20\/}~hex represents
the space.

\bigskip

\hfuzz.5pt\halign to\hsize{\tabskip3pt plus10pt
\hskip3pt\Strut\hfil\it#\hfil\quad\Vrule&&\hbox to.27in{\hfil\tt#\hfil}\cr
hex&\it0&\it1&\it2&\it3&\it4&\it5&\it6&\it7&\it8&\it9&\it A&\it B&\it C&\it D&\it E&\it F\cr
\noalign{\hrule height.5pt depth.5pt}
0x&NUL&SOH&STX&ETX&EOT&ENQ&ACK&BEL&BS&HT&LF&VT&FF&CR&SO&SI\cr
\noalign{\hrule}
1x&DLE&DC1&DC2&DC3&DC4&NAK&SYN&ETB&CAN&EM&SUB&ESC&FS&GS&RS&US\cr
\noalign{\hrule}
2x&\tt\char"20&!&"&\char"23&\char"24&\char"25&\char"26&'&(&)&*&+&,&-&.&/\cr
\noalign{\hrule}
3x&0&1&2&3&4&5&6&7&8&9&:&;&<&=&>&?\cr
\noalign{\hrule}
4x&@&A&B&C&D&E&F&G&H&I&J&K&L&M&N&O\cr
\noalign{\hrule}
5x&P&Q&R&S&T&U&V&W&X&Y&Z&[&\char"5C&]&\char"5E&\char"5F\cr
\noalign{\hrule}
6x&`&a&b&c&d&e&f&g&h&i&j&k&l&m&n&o\cr
\noalign{\hrule}
7x&p&q&r&s&t&u&v&w&x&y&z&\char"7B&\char"7C&\char"7D&\char"7E&DEL\cr
\noalign{\hrule}}

\bigskip\bigskip

The first 32 codes are assigned to functions rather than characters.
For example, {\tt HT} means ``move to next tab stop.'' Many of these
{\sl control codes\/} are for specific applications and rarely used
nowadays. Anyway, here is what they mean and their C escape sequences:

\bigskip

\hbox{
%  F I R S T   C O L U M N
\vtop{\hsize=0.50\hsize \halign{\tabskip12pt plus10pt%
\hfil\it#&\tt#\hfil&#\hfil\cr
\noalign{\bf Transmission Control}
01&SOH&Start of Heading\cr
02&STX&Start of Text\cr
03&ETX&End of Text\cr
04&EOT&End of Transmission\cr
05&ENQ&Enquiry (goes with {\tt ACK})\cr
06&ACK&Acknowledge\cr
10&DLE&Data Link Escape\cr
15&NAK&Negative Acknowledge\cr
16&SYN&Synchronous Idle\cr
17&ETB&End Transmission Block\cr
\noalign{\medskip\bf Device Control}
11&DC1&XON (okay to send)\cr
12&DC2&Device Control 2\cr
13&DC3&XOFF (pause listings)\cr
14&DC4&Device Control 4\cr
\noalign{\medskip\bf Code Extensions}
0E&SO&Shift Out (use alternate code)\cr
0F&SI&Shift In (resume default code)\cr
1B&ESC&Escape\cr}} \quad
%  S E C O N D   C O L U M N
\vtop{\hsize=0.46\hsize \halign{\tabskip12pt plus10pt%
\hfil\it#&\tt#\hfil&#\hfil&\tt#\hfil\cr
\noalign{\bf Formatting}
08&BS&Backspace&\string\b\cr
09&HT&Horizontal Tabulation&\string\t\cr
0A&LF&Line Feed&\string\n\cr
0B&VT&Vertical Tabulation&\string\v\cr
0C&FF&Form Feed (page eject)&\string\f\cr
0D&CR&Carriage Return&\string\r\cr
\noalign{\medskip\bf Information Separation}
1C&FS&File Separator\cr
1D&GS&Group Separator\cr
1E&RS&Record Separator\cr
1F&US&Unit Separator\cr
\noalign{\medskip\bf Others}
00&NUL&Null character&\string\0\cr
07&BEL&Bell (audible alert)&\string\a\cr
18&CAN&Cancel line\cr
19&EM&End of Medium\cr
1A&SUB&Substitute\cr}}
}

\nobreak\vskip0pt plus1filll
\rightline{\smallfont ujr/2000-10-06}

\bye
