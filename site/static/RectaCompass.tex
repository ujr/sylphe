% A4 paper: 210 x 297 mm
% Margins: 10mm on each side, 10mm between panels
% Therefore, each panel is 90 x 133.5 mm
% Contains specials for dvipdfm to include raster figures.
% Uses the epsf macro package to include the eps diagram.

\ifx\pdfoutput\undefined\csname newcount\endcsname\pdfoutput\fi
\ifcase\pdfoutput\def\pdfinfo#1{\special{pdf:docinfo<<#1>>}}\fi
\pdfinfo{ /Author (Recta (Rekonstruktion UJR 2013))
 /Title (Fl^^fcssigkeits-Kompass RECTA)
 /Subject (Gebrauchsanweisung Recta-Kompass) }

\hsize=190truemm
\vsize=277truemm
\voffset=-15.4truemm  % It seems dvi drivers use 1in
\hoffset=-15.4truemm  % we want 1cm, so subtract 15.4 from 1in
\newdimen\panelwd \panelwd=90truemm
\newdimen\panelht \panelht=133.0truemm
\newdimen\paneldblht \paneldblht=277.0 truemm
\nopagenumbers

\input epsf
\input german.sty
\input letterspacing
\parindent=1pc
\parskip=0pt plus 4pt
\hfuzz=.5ex % not fuzzy about right margins
\font\bigfont=cmr12 % headings
\font\tinyfont=cmr10 at 6pt
\font\smallfont=cmr8
\def\\{\hskip1ex plus 1ex minus .5ex\relax}
\def\extrawordstretch{%
 % Allow more word stretch than usual for CMR10
 % (2.22222pt) to help TeX with the short line width:
 \spaceskip=3.33333pt plus 3.33333pt minus 1.11111pt}
\def\beginpanel#1{\vbox to #1\bgroup
 \hsize=\panelwd\extrawordstretch}
\def\endpanel{\par\egroup}
\def\small{\smallfont\normalbaselineskip=9pt\normalbaselines}

\setbox1=\beginpanel\panelht
\topglue0pt\vfill
\vskip 125.5 truemm
\special{pdf:image width 88mm height 125.5mm (Titel.jpg)}
\vfill
\endpanel

\setbox8=\beginpanel\panelht
\topglue 0pt plus 1fill
\hrule
\bigskip
\centerline{\bigfont VORTEILE}
\smallskip
\centerline{\bigfont DES FL\"USSIGKEITS-KOMPASSES}
\smallskip
\centerline{\bigfont RECTA}
\bigskip
\parindent2em
\item{{\bf 1.}} Sehr schnelles Arbeiten im Felde durch
 Ausschaltung der Nadelschwingungen.
\medskip
\item{{\bf 2.}} Leichtes Bestimmen der genauen Richtung dank
 langer Ziellinie und der im Spiegel ersichtlichen Magnetnadel\-stellung.
\medskip
\item{{\bf 3.}} Grosse Leuchtkerben erleichtern die Verwendung im Dunkeln.
\medskip
\item{{\bf 4.}} Sinnreicher Bau, beim "Offnen geht der Spiegel
 auto\-matisch in die richtige Lage, ebenso gen"ugt eine
 Bewegung f"ur das Schliessen und das Zur"uckgehen des Spiegels.
\medskip
\item{{\bf 5.}} Geh"ause ganz aus Metall, solid, dauerhaft,
 praktisches Format.
\medskip
\item{{\bf 6.}} Kompasskapsel unzerbrechlich,
 durchsichtig und ver\-gr"ossernd.
\bigskip
\hrule
\vfill
{\tinyfont\baselineskip8pt\noindent
Digital restauriert und neu gesetzt anhand einer Gebrauchsanweisung
aus den 1950er Jahren, kurz bevor diese vollst"andig zerfiel.
\hfil---ujr, Juli 2013\parfillskip=0pt\par}
}

\setbox2=\beginpanel\panelht \parindent=0pt
% Gently increase letter spacing for this all-caps title:
\centerline{\bigfont\letterspace spread 1pc{GEBRAUCHSANWEISUNG}}
\medskip
\vskip 65truemm
\special{pdf:image width 88mm height 65mm (Fig1.jpg)}
\smallskip
\centerline{\bigfont Allgemeine Bemerkungen}
\medskip
1. {\bf Zum "Offnen} nehme man den Kompass in die linke Hand,
dr"ucke mit dem Daumen auf den Knopf~A und ziehe mit der rechten
Hand die Schnur~D. Der Spiegel~C stellt sich von selbst in die
richtige Lage. Durch Neigen des Kompasses l"asst sich der Spiegel
ins Geh"ause versenken und wieder hervorbringen.
\smallskip
2. {\bf Zum Schliessen} dr"ucke man mit Daumen und Zeige\-finger
auf die Kn"opfe~BB' und stosse den Kompass in das Geh"ause.
Ist der Spiegel vorher versenkt worden, so schliesse man
trotz des Widerstandes.
\smallskip
%3. {\bf F"ur das Anvisieren im Felde} ist der Spiegel zu verwen\-den.
3. {\bf F"ur das Anvisieren im Felde} den Spiegel verwenden.
Die N-Spitze (radiumisiert) der Nadel~E muss sich immer
\parfillskip=0pt % force last line to full length
\endpanel

\setbox3=\beginpanel\panelht \parindent=0pt
zwischen den Leuchtstrichen~L befinden. Der Kompass soll
horizontal gehalten werden, damit die Nadel frei schwimmt.
Der gesuchte Punkt wird sich immer in der durch
die Ziellinie G--G' gegebenen vertikalen Ebene befinden.
Durch Verwendung des Kompasses wie ein Senklot, indem
man ihn frei an der Schnur h"angen l"asst, lassen sich ober- und
unterhalb des Horizonts befindliche Punkte mit Genau\-igkeit bestimmen.
\smallskip
4. {\bf Beim Arbeiten auf der Karte} m"ussen die N-S Striche
des Zifferblattes parallel zu den N-S Strichen der Karte liegen.
Die Vorderseite~M des Kompasses muss immer dem Standort zugekehrt sein.
\smallskip
5. {\bf Einteilung des Zifferblattes:}\hfil\break
a) \ in  $360^\circ$\kern-.5ex, d.h.~72 Teilungen von $5^\circ$\hfil\break
b) \ in 6400 Artilleriepromille, d.h.~64 Teilungen.\hfil\break
\phantom{b) \ }(100 Artilleriepromille = $5^\circ$ $37'$ $30''$)
\smallskip
6. {\bf Deklination:} Der magnetische Pol stimmt nicht mit dem
geographischen Pol "uberein. Der Winkelabstand zwi\-schen den beiden
Polen ist die Deklination. Sie ver"andert sich mit der Zeit und im
Raum, von einem Jahr zum andern und von einem Lande zum andern.
Der Recta-Kompass gestattet die Einstellung auf diese Ver"anderung.
\smallskip
7. {\bf Bemerkungen.} {\small
\spaceskip 2.83337pt plus 2.83337pt minus 0.94446pt
Der Kompass soll nicht in der N"ahe
von eisernen Gegenst"anden (Helm, Pistole, Pickel) gebraucht werden;
von eisernen Gel"andern mindestens 25\kern.5ex m Abstand nehmen. $\diamond$
In der Fl"ussig\-keit k"onnen kleine Blasen entstehen, die ohne
Einfluss auf die Funktion des Instrumentes sind. Sie erscheinen
und verschwinden, hervorgerufen durch Ver"anderungen im Luftdruck
und in der Tem\-pe\-ra\-tur (H"ohen\-wechsel). $\diamond$
Die Leuchtkraft des Radiums kann durch Aussetzen in volles Licht
erh"oht werden.\par}
\smallskip
8. {\bf Achtung vor Reibungselektrizit"at!}\hfil\break
{\small Die Kapsel vor oder beim Gebrauch nicht trocken abreiben.\par}
\endpanel

\setbox45=\beginpanel\paneldblht
\centerline{\bigfont Arbeiten im Gel"ande}
\bigskip
\item{A.} {\bf Orientieren der Karte.}
\smallskip
\item{a)} Versenken des Spiegels~C.
\item{b)} Den Kompass so auf die Karte legen, dass die N-S Striche
 des Zifferblattes mit den N-S Strichen der Karte parallel liegen.
\item{c)} Karte und Kompass drehen bis N-Spitze der Nadel zwischen
 Leuchtstrichen~L liegt.
\smallskip
\centerline{\it oder}
\smallskip
\item{a')} Das N des Zifferblattes zum Index H des Kompasses
 bringen vermittels drehen des Reifens~F.
\item{b')} Den Kompass parallel zu den N-S Strichen der Karte legen.
\item{c')} Karte und Kompass drehen bis N-Spitze der Nadel zwischen
 den Leuchtstrichen~L liegt.
\bigbreak
\item{B.} {\bf Bestimmen der Marschrichtung.}
\smallskip
\item{a)} Versenken des Spiegels.
\item{b)} Auf der Karte die Marschrichtung (Richtungsachse)
 bestimmen durch Verbinden des Standortes~A mit dem Bestimmungsort
 B, C oder~D entweder unter Zuhilfenahme der L"angsseite des Kompasses
 (siehe A--B) oder eines mit Bleistift eingezeichneten Striches
 (siehe A--C) oder der Schnur (siehe A--D).
\item{c)} Den Kompass parallel zur Richtungsachse auf die Karte legen,
 Vorderseite~M dem Standort zugekehrt.
\item{d)} Den Reifen~F drehen bis die N-S Striche des Zifferblattes
 parallel mit den N-S Strichen der Karte liegen.
\item{e)} Den Kompass mit ausgezogenem Spiegel in Augenh"ohe bringen.
 Sich so weit drehen, bei gleichzeitiger Beobach\-tung der Nadel im Spiegel,
 bis die N-Spitze der Nadel zwischen den Leuchtstrichen~L liegt. Nun wird
 die Marschrichtung, eventuell ein erster Punkt, "uber die Ziellinie~G
 visiert.
\smallskip
\vskip 60truemm
\special{pdf:image width 88mm height 60mm (Fig2.jpg)}
\bigskip
\item{C.} {\bf Bestimmen eines im Gel"ande sichtbaren Punktes.}
\smallskip
\item{a)} Auf der Karte den (eigenen) Standort ermitteln.
\item{b)} Mit dem Kompass den zu bestimmenden Punkt anvisie\-ren,
 den Reifen~F drehen bis die N-Spitze der Nadel im Spiegel zwischen
 den Leuchtstrichen~L liegt.
\item{c)} Spiegel einschieben.
\item{d)} Kompass auf die Karte legen, L"angsseite an den
 Stand\-ort, Kompass um den Standort drehen bis die N-S Striche des
 Zifferblattes parallel zu den N-S Strichen der Karte liegen.
\item{e)} Der zu bestimmende Punkt liegt nun in der durch die
 L"angsseite des Kompasses gegebenen Linie.
\endpanel

\setbox67=\beginpanel\paneldblht
\item{D.} {\bf Bestimmen eines unbekannten Standortes.}
\smallskip
\item{a)} Auf der Karte zwei vom Standorte aus sichtbare,
 be\-kann\-te Punkte ermitteln.
\item{b)} Einen dieser Punkte anvisieren und durch Drehen
 des Reifens~F die N-Spitze der Nadel zwischen die Leucht\-striche bringen.
\item{c)} Spiegel einschieben.
\item{d)} Kompass auf die Karte legen, L"angsseite an den
 anvisier\-ten Punkt, Kompass um den Punkt drehen, ohne Drehen des
 Reifens~F, bis die N-S Striche des Zifferblattes parallel zu den
 N-S Strichen der Karte liegen.
\item{e)} Eine Linie vom anvisierten Punkt ausgehend parallel
 zur L"angsseite des Kompasses einzeichnen.
%\item{f)} Den zweiten Punkt anvisieren und gleich verfahren
% wie f"ur b--c.
\item{f)} Mit dem zweiten Punkt gleich verfahren wie f"ur b--e.
\item{g)} Der Schnittpunkt der zwei Linien gibt den gesuchten
 Standort an (je n"aher der Winkel zwischen den zwei Linien bei
 $90^\circ$ liegt, desto genauer l"asst sich der Standort ermitteln).
\bigbreak
\item{E.} {\bf Umgehung eines Hindernisses.}
\item{} (Grundsatz des gleichseitigen Dreiecks.)
\smallskip
\centerline{\epsfbox{Fig3.eps}}
\item{a)} Zur Umgehung eines Hindernisses Richtung N-Spitze der
 Nadel zwischen Umgehungspunkten K oder~K' nehmen, Schritte z"ahlen.
\item{b)} Sobald das Gel"ande es gestattet, in der urspr"unglichen
 Richtung N-Spitze der Nadel zwischen Leuchtstrichen weitergehen.
\item{c)} Nach "Uberschreiten des Hindernisses Richtung N-Spitze
 der Nadel zwischen den anderen Umgehungspunkten nehmen, die gleiche
 Anzahl Schritte wie unter a) mar\-schie\-ren.
\item{d)} Wieder urspr"ungliche Richtung N-Spitze der Nadel
 zwischen Leuchtstrichen einschlagen.
\bigbreak
\item{F.} {\bf Bestimmen des Azimuts.}\hfil\break
 Das Azimut ist der am Standort eingeschriebene Winkel
 zwischen den Linien zum geographischen Norden und zum anvisierten Punkt.
\smallskip
\item{a)} {\bf Auf der Karte} kann es mit dem Winkelmass,
 aber auch mit dem Kompass ermittelt werden, indem wie unter B (a--d)
 verfahren wird. Das Azimut ist dann beim Index~H ablesbar.
\smallskip
\item{b)} {\bf Im Gel"ande} ist das Azimut ablesbar beim Index~H,
 wenn das Ziel anvisiert ist und die N-Spitze der Nadel zwischen den
 Leuchtstrichen liegt. Mit Hilfe des Azimuts ist es leicht, von der
 Linie N--S oder O--W ausgehend, eine topographische Skizze anzufertigen.
\bigbreak
\item{G.} {\bf Regulieren auf die Deklination}\hfil\break
(soll nur vom Spezialisten ausgef"uhrt werden).\par{\small
\item{a)} Die Befestigungsschraube des Zifferblattes sorgf"altig
 um ungef"ahr eine halbe Drehung l"osen.
\item{b)} Das Zifferblatt auf die Deklination einstellen.
\item{c)} Schraube wieder sorgf"altig anziehen.\par}
% Need \par within the \small group or bad line spacing results
\endpanel

% Vorderseite: four panels
\topglue0pt\vfill
\line{\box1\hfil\box8}
\vskip 3truemm plus 1fill
\hrule
\vskip 3truemm plus 1fill
\line{\box2\hfil\box3}
\vfill\eject

% R"uckseite: two long columns
\topglue0pt\vfill
\line{\box45\hfil\box67}
\vfill\eject

% Show the tenrm word skip (can override with \spaceskip):
%\showthe\fontdimen2\tenrm % normal interword spacing
%\showthe\fontdimen3\tenrm % interword stretch (\spaceskip plus)
%\showthe\fontdimen4\tenrm % interword shrink (\spaceskip minus)

\bye
